\interlude{State-disruption resistance\\[1.1em]\Large{}Even more attack-worthy mitigations}

\begin{frame}{State Disruption Attacks}
  \only<1>{
    \centering
    An attacker, trying to prevent a successful key exchange
    \\[1.3em] by exploiting flaws in the protocol state machine.
    \\[1.3em] \textbf{A protocol-level denial of service.}
  } 
  \only<2>{
    \begin{itemize}
      \item Attacker wants to mount a protocol-level DOS attack
      \item Attacker may observe messages
      \item Attacker may insert messages, but they may not drop or modify messages
      \item Halfway between an active and passive attacker:
        \begin{itemize}
          \item For a fully active attacker state disruption is trivial; they can just drop messages
        \end{itemize}
    \end{itemize}
  }
\end{frame}

\begin{frame}{Non-Interruptability: More Formally}
    For every pair of traces $\mathtt{tmin}, \mathtt{tmax}$ where trace $\mathtt{tmax}$ can be formed by
    insertion of messages/oracle calls into $\mathtt{tmin}$, the result of $\mathtt{tmin}$ and $\mathtt{tmax}$
    should remain the same.

    \begin{itemize}
      \item Let $\mathtt{Result}$ be the set of possible protocol results
      \item Let $\mathtt{Trace}$ be the set of possible protocol traces
      \item Let $\mathtt{res}(\mathtt{t}) : \mathtt{Trace} \to \mathtt{Result}$ determine the protocol result given $\mathtt{t} : \mathtt{Trace}$
      \item Let $\mathtt{t1} \supseteq \mathtt{t2} : \mathtt{Trace} \to \mathtt{Trace} \to \mathtt{Prop}$ denote that $\mathtt{t}2$ can be formed by insertion of elements into $\mathtt{t}1$
      \item $\forall (\mathtt{tmin}, \mathtt{tmax}) : \mathtt{Trace} \times \mathtt{Trace}; \mathtt{tmin} \supseteq \mathtt{tmax} \to \mathtt{res}(\mathtt{tmin}) = \mathtt{res}(\mathtt{tmax})$
    \end{itemize}
    \pdfpcnote{Karo}
\end{frame}

\interlude[2]{The ChronoTrigger attack\\[1.1em]\Large{}DoS by exploiting retransmission protection}

\begin{frame}{Retransmission Protection in WireGuard}
  \small
  \begin{columns}[fullwidth,T]
    \begin{column}{.5\linewidth}
      \vspace{-.05\textheight}
      \rlap{\includegraphics[height=.90\textheight,page=3,clip,trim=0 20 0 50]{graphics/scientific/rosenpass-wireguard-attack-types.pdf}}
    \end{column}%
    \begin{column}{.46\linewidth}
      \small
      \begin{itemize}
        \item Replay attacks thwarted by counter
        \item Counter is based on real-time clock
        \item Responder is semi-stateful (one retransmission at program start may be accepted, but this does not affect protocol security)
        \item[$\Rightarrow$]
          WG requires \emph{either} reliable real-time clock \emph{or} stateful initiator
        \item[$\Rightarrow$]
          Adversary can attempt replay, but this cannot interrupt a valid handshake by the initiator
      \end{itemize}
    \end{column}
  \end{columns}
\end{frame}

\begin{frame}{ChronoTrigger Attack: Delayed Execution}
  \small
  \begin{columns}[fullwidth,T]
    \begin{column}{.5\linewidth}
      \vspace{-.05\textheight}
      \rlap{\includegraphics[height=.89\textheight,page=5,clip,trim=0 20 0 50]{graphics/scientific/rosenpass-wireguard-attack-types.pdf}}
    \end{column}

    \begin{column}{.46\linewidth}
      \only<+|handout:+>{
        \begin{enumblock}{Preparation phase:}
          \begin{enumerate}
            \item \textbf{Attacker} sets \emph{initiator system time} to a future value
            \item \textbf{Attacker} records \emph{InitHello} as \emph{KillToken} while both peers are performing a valid handshake
          \end{enumerate}
        \end{enumblock}
        \begin{enumblock}{Delayed execution phase:}
          \begin{enumerate}
            \item \textbf{Attacker} sends \emph{KillToken} to responder, setting their timestamp to a future value
            \item[$\Rightarrow$] Initiation now fails again due to timestamp mismatch
          \end{enumerate}
        \end{enumblock}
      }

      \only<+|handout:+>{%
        \leavevmode\begin{block}{Attacker gains}
          \begin{itemize}
            \item Extremely cheap protocol-level DOS
          \end{itemize}
          \unskip
        \end{block}

        \begin{block}{Preparation phase, attacker needs:}
          \begin{itemize}
            \item Eavesdropping of initiator packets
            \item Access to system time
          \end{itemize}
          \unskip
        \end{block}

        \begin{block}{Delayed execution, attacker needs:}
          \begin{itemize}
            \item No access beyond message transmission to responder
          \end{itemize}
          \unskip
        \end{block}

        }

        \only<+|handout:+>{%

          \begin{block}{Gaining access to system time:}
            \begin{itemize}
              \item Network Time Protocol is insecure,\\
                Mitigations are of limited use
              \item[$\Rightarrow$] Break NTP \emph{once}; kill token lasts forever
            \end{itemize}
            \unskip
          \end{block}
          }
    \end{column}
  \end{columns}
\end{frame}

\begin{frame}{Fixing ChronoTrigger}
  \centering
  Rosenpass uses a stateless-responder
  \\[1.3em] so there is no state to disrupt.
\end{frame}

\interlude[3]{The CookieCutter attack\\[1.1em]\Large{}DoS by exploiting proof-of-ip-ownership}

\begin{frame}{Proof-of-IP ownership in WireGuard}
  \textbf{CookieReply} (responder to initiator):
  \\[1.3em]\blockquote{I am under load. Prove you are not using an IP spoofing attack with this \emph{cookie key}.}
\end{frame}

\begin{frame}{CookieCutter Attack}
  \small
  \begin{columns}[fullwidth,T]
    \begin{column}{.4\linewidth}
      \vspace{-.05\textheight}
      \rlap{\includegraphics[height=.89\textheight,page=9,clip,trim=0 20 0 50]{graphics/scientific/rosenpass-wireguard-attack-types.pdf}}
    \end{column}
    \begin{column}{.47\linewidth}
      \only<1>{
        \begin{enumerate}
          \item \textbf{Attacker:} continuous DOS attack against responder
          \item \textbf{Initiator:} begin handshake, sends \emph{InitHello}
          \item \textbf{Responder:} sends \emph{CookieReply}
          \item \textbf{Initiator:} store cookie key \& waits for retransmission timer
          \item \textbf{Attacker:} forge a cookie reply with fake \emph{cookie key}
          \item \textbf{Initiator:} overwrites valid cookie key with the fake one
          \item[…] Repeat ad nauseam
        \end{enumerate}
      }
      \only<2>{
        \begin{block}{Attacker gains:}
          \begin{itemize}
            \item Cheap protocol-level DOS
          \end{itemize}
          \unskip
        \end{block}
        \begin{block}{Attacker needs:}
          \begin{itemize}
            \item Knowledge of public keys
            \item Good timing 
          \end{itemize}
          \unskip
        \end{block}
        \begin{block}{Role switching:}
          \begin{itemize}
            \item WireGuard sometimes uses role switching
            \item To account for that, the attack can be performed against both peers 
          \end{itemize}
          \unskip
        \end{block}
      }
    \end{column}
  \end{columns}
\end{frame}

\begin{frame}{Fixing CookieCutter}
  \centering
  Immediately retransmit InitHello on receiving CookieReply.
  \pause
  \\[1.6em] (Or find a better knock-pattern than proof-of-ip ownership.)
\end{frame}
