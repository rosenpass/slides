\documentclass{rosenpass-beamer}
\usepackage[german]{babel}

\usepackage{emoji}

\usepackage{xurl}
\urlstyle{same}

\homepage{\url{https://rosenpass.eu}}
\social{\url{https://chaos.social/@rosenpass}}

\conference{MRMCD 2023}
\date{2023-09-03}

\title{Rosenpass}
\subtitle{
  Sichere Kryptografie trotz Quantencomputern: Projektupdate
}

\author{Emil~Engler, \ Stephan~Ajuvo, \ Karolin~Varner}
\contributor{
  \relscale{.90}
  Marei~Peischl, Lisa~Schmidt, Steffen~Vogel, Alice~Bowman, Wanja~Zaeske, Sven~Friedrich, Benjamin~Lipp
}
\funding{Funding: NLNet \& Prototype Fund}

% https://tex.stackexchange.com/questions/2541/beamer-frame-numbering-in-appendix
\newcommand{\backupbegin}{
  \newcounter{framenumberappendix}
  \setcounter{framenumberappendix}{\value{framenumber}}
}
\newcommand{\backupend}{
  \addtocounter{framenumberappendix}{-\value{framenumber}}
  \addtocounter{framenumber}{\value{framenumberappendix}} 
}

\newcommand*{\outrobegin}{
  \setbeamercolor{footnoteblock}{bg=rosenpass-lightorange,fg=black}
  \tcbset{frame-number/.style={colback=rosenpass-lightorange,frame empty,top=.25ex,bottom=.25ex,left=.25ex,right=.25ex,colupper=rosenpass-gray,fontupper=\usebeamerfont{frame number},before=,after=,box align=base}}
}

\newcommand*{\outro}{
  \nonumbernote{
      rosenpass.eu | @rosenpass@chaos.social | info@rosenpass.eu
      \scriptsize
      \\ Emil Engler @engler@chaos.social | Stephan Ajuvo @ajuvo@chaos.social | Karolin Varner @kora@chaos.social
  }
}

% https://tex.stackexchange.com/questions/404429/align-beamer-color-box-at-bottom-of-frame
\newcommand{\nonumbernote}[1]{%
  {%
      \setbeamertemplate{footnote}{%
        \parindent 1em\noindent%
        \raggedright
        \insertfootnotetext\par%
    }
    \footnotetext[42]{#1}
  }
}


% reduce itemize indent
\setlength{\leftmargini}{0pt}

\usepackage{biblatex}
\addbibresource{sources.bib}
\graphicspath{{}{graphics/}}

\begin{document}

\maketitle

		
\begin{frame}{Was passiert im Talk?}
\begin{itemize}
  \item Was bisher geschah
  \item Zusammenfassung vom EH20 Talk: Was ist Rosenpass
  \item Was nach dem Easterhegg passiert ist
  \item Was wir nun vor haben
    \begin{itemize}
      \item go-rosenpass
      \item NetBird
      \item Broker-Architektur \& Schnittstellen zum Einbinden
    \end{itemize}
\end{itemize}
\end{frame}

\begin{frame}{Was bisher geschah}
  \begin{itemize}
    \item Seit 2020: Entwicklung der Kryptografie \& der Software
    \item Feb. 2023: Softwarerelease \& Whitepaper
    \item März 2023: NLNet Projekt um Sicherheitsbeweis mit CryptoVerif zu erzeugen
    \item März 2023: Talk auf dem Real World Post-Quantum Krypto Workshop in Tokyo
    \item April 2023: Vorstellung/Erklärung auf dem Easterhegg\footnote{\url{https://media.ccc.de/search/?q=rosenpass}}
    \item Aug. 2023: Release Kandidat 0.2.0 mit FreeBSD unterstützung
    \item Sep. 2023: Beginn des Prototype Fund 14 Projektes für Isolation in Rosenpass
  \end{itemize}
\end{frame}

\begin{frame}{Warum sind Quantencomputer (k)eine Bedrohung?}
\begin{columns}[b]
\begin{column}{.75\textwidth}
\begin{itemize}
  \item Grovers Algorithmus \strong{schwächt} symmetrische Kryptografie
  \begin{itemize}
    \item AES, SHA-2, SHA-3, Chacha20
    \item Lösung: größere Keys
  \end{itemize}
  \item Shors Algorithmus \strong{bricht} asymmetrische Kryptografie
  \begin{itemize}
    \item RSA, DSA, DH, ECDH
    \item Lösung: alternative Kryptografie
  \end{itemize}
  \item  Nur auf großen Quantencomputern
  \begin{itemize}
    \item Die existieren noch nicht
    \item Problem: Store now, decrypt later
  \end{itemize}
\end{itemize}
\end{column}
\begin{column}{.25\textwidth}
\makebox[\linewidth][r]{%
\includegraphics[width=1.5\linewidth]{graphics/qc and crypto.png}
}%
\par
\imgNote{\makebox[\linewidth][r]{Quantencomputer überschatten Kryptografieverfahren.}}
\end{column}
\end{columns}
\end{frame}
		
\begin{frame}{PQ-sichere VPNs: WireGuard + Rosenpass}
  \begin{itemize}
    \item  Hybride Sicherheit

    \begin{itemize}
      \item Bricht nur, wenn Rosenpass \textbf{und} WireGuard versagen
    \end{itemize}
    \item  Überall nutzbar, wo WireGuard schon läuft
    \item  Ohne Anpassung vom WireGuard Source Code

    \begin{itemize}
      \item Shared Secret aus Rosenpass = PSK für WireGuard
    \end{itemize}
    \item  Aber:

    \begin{itemize}
      \item Ein Prozess mehr
      \item Handshake alle 2 Minuten
    \end{itemize}
  \end{itemize}

\makebox[\linewidth][c]{%
\includegraphics[width=.6\linewidth]{graphics/wireguard and rp.png}}

\imgNote{\makebox[\linewidth][c]{WireGuard mit Rosenpass.}}
\end{frame}

\begin{frame}{Rosenpass: Sicherheitseigenschaften}

\vspace{0.5em}
\begin{columns}[t]
\begin{column}{.30\textwidth}
\heading{WireGuard}
\begin{itemize}
  \itemtick Session-key secrecy
  \itemtick \dots
  \itemtick Identity Hiding
  \itemfail \textbf{Non-Interruptability} \footnote[frame]{Angenommen der Systemzeit wird Vertraut}
  \itemfail \textbf{Post-Quantum Security}
\end{itemize}
\end{column}

\begin{column}{.30\textwidth}
\heading{
  PQ WireGuard
  \footnote[frame]{
	  Hülsing, Ning, Schwabe, Weber, Zimmermann. “Post-quantum WireGuard”. https://ia.cr/2020/379
	}
}
\begin{itemize}
  \itemtick \textbf{Post-Quantum Security}
  \itemfail \textbf{Hybrid security}
  \itemfail \textbf{Non-Interruptability} \footnote[frame]{Assuming a PSK}
\end{itemize}
\end{column}

\begin{column}{.30\textwidth}
\heading{Rosenpass}
\begin{itemize}
  \itemtick \textbf{Non-Interruptability} \footnote[frame]{Through cookies}
  \itemtick \textbf{Hybrid security} \footnote[frame]{Wenn es mit WireGuard benutzt wird}
\end{itemize}
\end{column}

\end{columns}
\vspace{1.5em}

\end{frame}

\begin{frame}{Zum Nachbauen… aus dem Whitepaper:}
  \includegraphics[height=.9\textheight]{graphics/rosenpass-wp-message-handling-code.pdf}
\end{frame}

\begin{frame}{Zum Nachbauen… go-rosenpass – Steffen Vogel FTW}
  \includegraphics[height=.9\textheight]{assets/2023-09-02-go-rosenpass-gh.png}
\end{frame}

\begin{frame}{Zum Nachbauen… go-rosenpass – Steffen Vogel FTW}
  \includegraphics[height=.9\textheight]{assets/2023-09-02-steffen-proof-read.png}
\end{frame}

\begin{frame}{Zum Nachbauen… go-rosenpass – Steffen Vogel FTW}
  \includegraphics[height=.9\textheight]{assets/2023-09-02-steffen-proof-read-issues.png}
\end{frame}

\begin{frame}{Zum Integrieren… NetBird}
  \includegraphics[height=.9\textheight]{assets/2023-09-02-netbird-gh.png}
\end{frame}

\begin{frame}{Zum Integrieren… NetBird}
  \includegraphics[height=.9\textheight]{assets/2023-09-02-netbird-gh-rosenpass-testemonial.png}
\end{frame}

\begin{frame}{Zum Integrieren… NetBird}
\begin{columns}[c]
\begin{column}{0.7\textwidth}
  \begin{itemize}
    \item Netbird: Einfaches user interface
    \item Rosenpass: Hochsichere PQ-Crypto
    \item
      go-rosenpass: Um Plattformen zu unterstützen auf denen die Rust Variante schwer zu integrieren ist
      \begin{itemize}
        \item Android
        \item iOS
        \item Windows
        \item \dots
      \end{itemize}
  \end{itemize}
\end{column}

\begin{column}{0.2\textwidth}
\includegraphics[width=\linewidth]{graphics/rosenpass in anderen apps.png}

\medskip
\includegraphics[width=\linewidth]{graphics/Illu-install.png}

\end{column}
\end{columns}
\end{frame}

\begin{frame}{Zum Integrieren… Prototypefund 14 Projekt}
\begin{columns}[c]
\begin{column}{0.7\textwidth}
  \begin{itemize}
    \item Schnittstelle zwischen Komponenten
    \item Kommunikation über Unix-Sockets
    \item Spezielle Serialisierungsbibliothek für Schlüsseldaten\footnotemark
    \item Broker-Pattern für Rosenpass– Jede Komponente in einem eigenen Prozess
    \item Mikro-VMs um wirklich hohe sicherheit zu haben
    \item Minimale Privilegien; Sandboxing
  \end{itemize}
\end{column}

\begin{column}{0.2\textwidth}
\includegraphics[width=\linewidth]{graphics/rosenpass in anderen apps.png}

\medskip
\includegraphics[width=\linewidth]{graphics/Illu-install.png}

\end{column}
\end{columns}

\footnotetext{Externes Memory-Management}

\end{frame}

\outrobegin

\begin{frame}[fragile]{Rosenpass Roadmap}
\begin{columns}[c]
\begin{column}{0.7\textwidth}
\begin{itemize}
  \item Sicherheitsbeweis
  \item Formelle Verifikation der Implementierung
  \item Einfachere Benutzbarkeit

  \item  Kryptografie + Safety Forschung:
  \begin{itemize}
    \item Kryptografie in der Avionik
    \item Decryption Despite Error
  \end{itemize}
\end{itemize}
\end{column}

\begin{column}{0.2\textwidth}
\includegraphics[width=\linewidth]{graphics/rosenpass in anderen apps.png}

\medskip
\includegraphics[width=\linewidth]{graphics/Illu-install.png}

\end{column}
\end{columns}

\outro

\end{frame}

\begin{frame}{Rosenpass Strukturpläne}
\begin{columns}[c]
\begin{column}{0.7\textwidth}
\begin{itemize}
  \item Translationsforschung: Schnittstelle zwischen Industrie und Wissenschaft
  \item Mit mehreren Integratoren arbeiten; Open-Source R\&D-Abteilung
  \item Antihierarchisches Arbeiten
  \item Karo hätte gerne mal wieder Freizeit
\end{itemize}
\end{column}

\begin{column}{0.2\textwidth}
\includegraphics[width=\linewidth]{graphics/rosenpass in anderen apps.png}

\medskip
\includegraphics[width=\linewidth]{graphics/Illu-install.png}

\end{column}
\end{columns}

\outro

\end{frame}

\backupbegin

\begin{frame}{Idee: Kryptografie als Notar Erklären}
\begin{columns}[c]

\begin{column}{0.7\textwidth}
  \begin{itemize}
    \item Problem: Kryptografie wird als schwarze Magie verstanden
    \item Problem: Krude vorschläge zur Verwaltungsautomatisierung
    \item Problem: Und Strafverfolgung
    \item Problem: Krypto wird auf Verschlüsselung reduziert
    \item
      Problem: Kaum jemand weiß was moderne Verfahren tun
      \begin{itemize}
        \item Elliptic-Curve Pairings
        \item Multi-Party Computation
        \item Homomorpe Verschlüsselung
        \item Datenbanken mit anonymem Zugriff
        \item Anonyme Kommunikation
      \end{itemize}
  \end{itemize}
\end{column}

\begin{column}{0.2\textwidth}
  \ImgSource{
    \includegraphics[height=.5\textheight]{assets/rabbit-with-clock.png}
  }{White Rabbit aus Alice in Wonderland – CC0}
  \imgNote{Häschen sind die besseren Menschen.}
\end{column}

\end{columns}

\outro

\end{frame}

\begin{frame}{Idee: Kryptografie als Notar Erklären}
\begin{columns}[c]

\begin{column}{0.7\textwidth}
  \begin{itemize}
    \item Krypto: Einsatzfähig für viele Prozesse in denen Information Übertragen wird
    \item Datenschutz: Anonyme, Nutzergesteuerte Prozesse
    \item Idee: Metapher von Kryptografie als Notar
      \begin{itemize}
        \item Notare werden Bestraft wenn sie Dinge Zusichern die sie nicht können
        \item Oder wenn sie gegen Regeln verstoßen
        \item Spezieller schutz vor dem Recht
        \item Kryptografie: Ähnlich, nur Matematisch, statt mit Staatsgewalt
      \end{itemize}
  \end{itemize}
\end{column}

\begin{column}{0.2\textwidth}
  \ImgSource{
    \includegraphics[height=.5\textheight]{assets/rabbit-with-clock.png}
  }{White Rabbit aus Alice in Wonderland – CC0}
  \imgNote{Häßchen sind die besseren Menschen.}
\end{column}

\end{columns}
\outro
\end{frame}

\backupend

\end{document}
