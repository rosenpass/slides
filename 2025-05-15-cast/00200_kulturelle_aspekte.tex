\interlude[0]{Safety \& Security \\ \Large \hrule \vspace{1.2em} Kulturelle Aspekte}
\section{Safety \& Security}

\begin{frame}[T]{Safety \& Security: Definition}
% GOAL Beide Begriffe einführen
\small
  \begin{columns}[t,fullwidth]
   \hfill
    \begin{column}{.45\linewidth}
      \begin{block}{Safety}
      \begin{itemize}
        \item Schutz von Lebewesen
        \item Zumeist Cyber-Physisch, z.B. Computer + Elektromechanischer Aktuator
      \end{itemize}
      \end{block}
    \end{column}
    \hfill
    \begin{column}{.45\linewidth}
      \begin{block}{Security}
      \begin{itemize}
        \item Schutz von Informationen
        \item Entweder: Information wird Dritten Zugänglich
        \item Oder: Information wird durch Dritte Manipuliret
      \end{itemize}
      \end{block}
    \end{column}
    \hfill
  \end{columns}
\end{frame}
% TODO Definitions slide für Safety und Security hinzufügen

\begin{frame}[T]{Problemstellungen}
% GOAL Verfahren sind beim ersten Hinblick inkompatibel, Klar machen, dass beide Begriff verschieden sind und verschiedene Vorgehen erfordern
% Folien 6, 8,: Problemstellung
% Folie 9: Es gibt gründe für die Unvereinbarkeit
% Folie 7: Es gibt aber auch gemeinsamkeiten
% TODO Grafik aus Poster einfügen
\small
  \begin{columns}[t,fullwidth]
   \hfill
    \begin{column}{.45\linewidth}
      \begin{block}{Safety}
      \begin{itemize}
        \item Zufällige Fehler
        \item Im Fehlerfall: Weiterbetrieb ermöglichen!
        \item Stabile Zieldefinition: Physik bleibt gleich
        \item Abgehangene Software -> Stabile Software!
        \item Rigide Validierungsprozesse
      \end{itemize}
      \end{block}
    \end{column}
    \hfill
    \begin{column}{.45\linewidth}
      \begin{block}{Security}
      \begin{itemize}
        \item Intelligente Angreifer
        \item Im Fehlerfall: Lieber das System stoppen
        \item Zieldefinition ist in Bewegung: Angreifer lernen auch dazu
        \item Abgehangene Software -> CVEs bekannt?
        \item Viele Freiheitsgrade in Validierung
      \end{itemize}
      \end{block}
    \end{column}
    \hfill
  \end{columns}
\end{frame}

%
\begin{frame}[T]{Akzeptanzkriterien}
  % GOAL erklären, wie man in Safety/Security Konfidenz erzeugt, einen guten Job gemacht zu haben
  % TODO make this a table
  \begin{itemize}
    \item Safety
    \begin{itemize}
      \item Goldstandard: Requirements-getriebenes Testen % üblich
      \item Alternative: "Proven in use" % unüblich
      \item Alternative: Formale Verifikation % selten
    \end{itemize}

    \item Security
    \begin{itemize}
      \item "Proven in use" reicht nicht, aber neuen Verfahren wird dennoch misstraut
      \item Testen unzureichend, aufgrund gezielter Angriffe
      \item Formale Verifikation ist der Goldweg
    \end{itemize}

  \end{itemize}
\end{frame}

\begin{frame}[T]{Kulturen}
  % GOAL Es gibt Unterschiede, die Begründet sind. Wenn Safety und Security verbunden werden sollen, müssen beide Gründe verstanden werden.
  \begin{itemize}
    \item Safety
    \begin{itemize}
      \item Menschen Sterben bei Versagen
      \item Probleme sind Verstanden und Stabil
      \item => Konservative Ingenieurskultur
    \end{itemize}

    \item Security
    \begin{itemize}
      \item Versagen erzeugt eher Finanziellen Schaden
      \item Problemtypen sind dynamisch und ändern sich dauernd
      \item => Progressive Ingenieurskultur
    \end{itemize}

    \item Security und Safety Kombiniert:
    \begin{itemize}
      \item Menschen sterben bei Versagen
      \item Aber die Probleme sind dynamisch, die Zielsetzung in Bewegung
      \item Konservativ \unicode{x21af} Progressiv
    \end{itemize}
  \end{itemize}
\end{frame}

\begin{frame}[T]{Versöhnung} % TODO more scientific term
  % GOAL Klar machen, dass beide Domänen nicht unversöhnbar sind
  \begin{itemize}
    \item Hohes Zuverlässigkeit nötig
    \item Analyse von Softwaresystemen in reeller Hardware
    \item Rigorose Validierungsprozesse (Peer-Review)
    \item Klarheit der Systemziele ist essentiell
    \item Umsetzung redundanter Systeme verbreitet
  \end{itemize}
\end{frame}
