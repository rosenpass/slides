\interlude[0]{Safety \& Security \\ \Large \hrule \vspace{1.2em} Kulturelle Aspekte}
\section{Safety \& Security}

\begin{frame}[T]{Safety \& Security in Computersystemen}
% GOAL Beide Begriffe einführen
\small
	% TODO Grafik einfügen
  \begin{columns}[t,fullwidth]
   \hfill
    \begin{column}{.45\linewidth}
      \begin{block}{Safety}
      \begin{itemize}
        \item Schutz von Lebewesen
      % \begin{itemize}
      % \end{itemize}

      \end{itemize}
      \end{block}
    \end{column}
    \hfill
    \begin{column}{.45\linewidth}
      \begin{block}{Security}
      \begin{itemize}
        \item Schutz von Informationen
        \begin{itemize}
          \item Information vor Dritten geheimhalten
          \item Information vor Manipulation durch Dritte schützen
        \end{itemize}
      \end{itemize}
      \end{block}
    \end{column}
    \hfill
  \end{columns}
\end{frame}

\begin{frame}[T]{Problemstellungen \& Rahmenbedingungen}
% GOAL Verfahren sind beim ersten Hinblick inkompatibel, Klar machen, dass beide Begriff verschieden sind und verschiedene Vorgehen erfordern
% TODO Grafik aus Poster einfügen? Problem: Graifk misch Problem, Kultur und Dialog. Grafik ist english.
% Tabelle anstatt grafik
\small
  \begin{columns}[t,fullwidth]
   \hfill
	% TODO make table
    \begin{column}{.45\linewidth}
      \begin{block}{Safety}
      \begin{itemize}
        \item Fehler: Zufällig
        \item Im Fehlerfall: Weiterbetrieb ermöglichen!
        \item Stabile Zieldefinition:\\Physik bleibt gleich
        \item Abgehangene Software $\rightarrow$ Stabil!
        \item Normierte Validierungsprozesse
      \end{itemize}
      \end{block}
    \end{column}
    \hfill
    \begin{column}{.45\linewidth}
      \begin{block}{Security}
      \begin{itemize}
        \item Fehler: Gezielt durch Angreifer
        \item Im Fehlerfall: Lieber das System stoppen
        \item Zieldefinition ist in Bewegung:\\Angreifer lernen dazu
        \item Abgehangene Software $\rightarrow$ Unsicher?
        \item Dynamische Validierungsprozesse
      \end{itemize}
      \end{block}
    \end{column}
    \hfill
  \end{columns}
\end{frame}


\begingroup
\ExplSyntaxOn
\definecolor{green1}{RGB}{48, 175, 155}
\definecolor{green2}{RGB}{130, 221, 207}
\colorlet{green3}{rosenpass-lightblue}
\newcommand*{\Level}[1]{
	\str_case:nnF {#1} {
		{+}{\cellcolor{green3}+}
		{++}{\cellcolor{green2}+\,+}
		{+++}{\cellcolor{green1}+\,+\,+}
		{-}{\cellcolor{red!10}}
	}
	{#1}
}
\ExplSyntaxOff

\begin{frame}[T]{Vertrauen schaffen: Akzeptanzkriterien}
  % GOAL erklären, wie man in Safety/Security Konfidenz erzeugt, einen guten Job gemacht zu haben
%  \begin{table}[]
    \begin{tabular}{l|cc|cc}
      \multirow{2}{*}{\bfseries Kriterium}
        & \multicolumn{2}{c|}{\bfseries Safety}
        & \multicolumn{2}{c}{\bfseries Security} \\ %TODO transpose
                           & \bfseries Konfidenz & \bfseries Verbreitung & \bfseries Konfidenz & \bfseries Verbreitung \\
    \hline
%     \\% TODO (also eigentlich Hinweis) Wanja: Hier ist eine Leerzeile die keine zweite/dritte Spalte hat, daher wird die vertikale linie nicht durchgezogen, aber vertikale linien sind eh doof
	% alternativ &&\\
    Praktische Tests          & \Level{+++}    & \Level{+++}     & \Level{+}           & \Level{++}       \\
    Proven-in-use             & \Level{++}     & \Level{+}       & \Level{-}           & \Level{+}        \\
    Mathematische Beweise     & \Level{+++}    & \Level{-}       & \Level{+++}         & \Level{++}       \\
    Externe Audits            & \Level{++}     & \Level{+++}     & \Level{+++}         & \Level{+++}      \\
    \end{tabular}
      \textbf{Verbreitung} Häufigkeit als tragendes Argument im Assurance-Case\\
      \textbf{Konfidenz} Vertrauen in das Kriterium


  % \begin{itemize}
  %   \item Safety
  %   \begin{itemize}
  %     \item Goldstandard: Requirements-getriebenes Testen % üblich
  %     \item Alternative: "Proven in use" % unüblich
  %     \item Alternative: Formale Verifikation % selten
  %   \end{itemize}

  %   \item Security
  %   \begin{itemize}
  %     \item "Proven in use" reicht nicht, aber neuen Verfahren wird dennoch misstraut
  %     \item Testen unzureichend, aufgrund gezielter Angriffe
  %     \item Formale Verifikation ist der Goldweg
  %   \end{itemize}
  % \end{itemize}

\end{frame}
\endgroup

\begin{frame}[T]{Ingenieurskulturen}
  % GOAL Es gibt Unterschiede, die Begründet sind. Wenn Safety und Security verbunden werden sollen, müssen beide Gründe verstanden werden.
	\begin{columns}[t,fullwidth]
		\hfill
		\begin{column}{.45\linewidth}
			\begin{block}{Safety $\Longrightarrow$ Konservativ}
				\begin{itemize}
				\item Menschen Sterben bei Versagen
				\item Probleme sind Verstanden und Stabil
				\end{itemize}
			\end{block}
		\end{column}
		\begin{column}{.45\linewidth}
			\begin{block}{Security $\Longrightarrow$ Progressive}
				\begin{itemize}
				\item Versagen erzeugt eher finanziellen Schaden
				\item Problemtypen sind dynamisch und ändern sich dauernd
				\end{itemize}
			\end{block}
		\end{column}
		\hfill
	\end{columns}

	% TODO marei center this block?
    \begin{block}{Security + Safety $\Longleftrightarrow$ Konservativ $\lightning$ Progressiv}
	    \begin{itemize}
	      \item Menschen sterben bei Versagen
	      \item Probleme sind dynamisch, Zielsetzung in Bewegung
	    \end{itemize}
  	\end{block}
\end{frame}

\begin{frame}[T]{Safety + Security: Checkliste}
  % GOAL Klar machen, dass beide Domänen nicht unversöhnbar sind
  \begin{enumerate}
	% TODO add checkmarks checkboxes für mehr Checklisten Vibe
	% TODO smallcaps
	% TODO checkbox right aligend
    \item Hohe Zuverlässigkeit
    \item Klarheit über Systemziele
    \item Umfassende Validierung
	\item Unabhängiges Review
    \item Analyse von Softwaresystemen in reeller Hardware
    \item Redundante Systeme
    \item \textbf{Kryptoagilität}
  \end{enumerate}
\end{frame}
