\interlude[0]{Safety \& Security \\ \Large \hrule \vspace{1.2em} Kulturelle Aspekte}
\section{Safety \& Security}

\begin{frame}[T]{Safety \& Security: Unterschiede}
\small
  \begin{columns}[t,fullwidth]
   \hfill
    \begin{column}{.45\linewidth}
      \begin{block}{Safety}
      \begin{itemize}
        \item Zufällige Fehler
        \item Systemausfall kann tödlich sein
        \item Stabile Zieldefinition: Physik bleibt gleich
        \item Abgehangene Software -> Stabile Software!
        \item Rigide Validierungsprozesse
      \end{itemize}
      \end{block}
    \end{column}
    \hfill
    \begin{column}{.45\linewidth}
      \begin{block}{Security}
      \begin{itemize}
        \item Intelligente Angreifer
        \item Im Fehlerfall: Lieber das System stoppen
        \item Zieldefinition ist in Bewegung: Angreifer lernen auch dazu
        \item Abgehangene Software -> CVEs bekannt?
        \item Viele Freiheitsgrade in Validierung
      \end{itemize}
      \end{block}
    \end{column}
    \hfill
  \end{columns}
\end{frame}


\begin{frame}[T]{Safety \& Security: Gemeinsamkeiten}
  \begin{itemize}
    \item Hohes Zuverlässigkeit nötig
    \item Analyse von Softwaresystemen in reeller Hardware
    \item Rigorose Validierungsprozesse
    \item -> limitieren minimale, maximale Validierungsgüte
  \end{itemize}
\end{frame}


\begin{frame}[T]{Methodiken}
  \begin{itemize}
    \item Sicherheit kann durch bewährte Praxis argumentiert werden, oft gilt „alt==Gold“
    \item Sicherheit kann nicht so argumentiert werden, der goldene Weg ist die formale Verifizierung
    \item => niemand behauptet, dass 3DES aufgrund seines Alters gut ist
    \item => aber auch neuartigen Algorithmen wird weniger vertraut...
    \item Sicherheit geht (oft) von zufälligem Versagen aus
    \item Sicherheit geht von gezielten Angriffen aus
    \item Sicherheit, die gegen entschlossene Angreifer gilt, gilt auch gegen zufälliges Versagen
    \item Redundanz funktioniert in beiden Bereichen
  \end{itemize}
\end{frame}
