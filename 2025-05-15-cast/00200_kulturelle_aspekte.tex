\interlude[0]{Safety \& Security \\ \Large \hrule \vspace{1.2em} Kulturelle Aspekte}
\section{Safety \& Security}

\begin{frame}[T]{Safety \& Security: Definition}
% GOAL Beide Begriffe einführen
\small
  \begin{columns}[t,fullwidth]
   \hfill
    \begin{column}{.45\linewidth}
      \begin{block}{Safety}
      \begin{itemize}
        \item Schutz von Lebewesen
      \begin{itemize}
         \item Maschine, die Computer enthält
      \end{itemize}
      \end{itemize}
      \end{block}
    \end{column}
    \hfill
    \begin{column}{.45\linewidth}
      \begin{block}{Security}
      \begin{itemize}
        \item Schutz von Informationen
        \begin{itemize}
          \item Information vor Dritten geheimhalten
          \item Information vor Manipulation durch Dritte schützen
        \end{itemize}
      \end{itemize}
      \end{block}
    \end{column}
    \hfill
  \end{columns}
\end{frame}

\begin{frame}[T]{Problemstellungen \& Rahmenbedingungen}
% GOAL Verfahren sind beim ersten Hinblick inkompatibel, Klar machen, dass beide Begriff verschieden sind und verschiedene Vorgehen erfordern
% TODO Grafik aus Poster einfügen? Problem: Graifk misch Problem, Kultur und Dialog. Grafik ist english.
\small
  \begin{columns}[t,fullwidth]
   \hfill
    \begin{column}{.45\linewidth}
      \begin{block}{Safety}
      \begin{itemize}
        \item Zufällige Fehler
        \item Im Fehlerfall: Weiterbetrieb ermöglichen!
        \item Stabile Zieldefinition:\\Physik bleibt gleich
        \item Abgehangene Software $\rightarrow$ Stabil!
        \item Normierte Validierungsprozesse
      \end{itemize}
      \end{block}
    \end{column}
    \hfill
    \begin{column}{.45\linewidth}
      \begin{block}{Security}
      \begin{itemize}
        \item Gezielte Fehler durch Angreifer
        \item Im Fehlerfall: Lieber das System stoppen
        \item Zieldefinition ist in Bewegung:\\Angreifer lernen auch dazu
        \item Abgehangene Software $\rightarrow$ Unsicher?
        \item Dynamische Validierungsprozesse
      \end{itemize}
      \end{block}
    \end{column}
    \hfill
  \end{columns}
\end{frame}

%
\begin{frame}[T]{Vertrauen schaffen: Akzeptanzkriterien}
  % GOAL erklären, wie man in Safety/Security Konfidenz erzeugt, einen guten Job gemacht zu haben

  % TODO marei vertical bar in table is weird
  \begin{table}[]
    \begin{tabular}{l|cc|cc}
      \multirow{2}{*}{\bfseries Kriterium}
        & \multicolumn{2}{c}{\bfseries Safety}
        & \multicolumn{2}{c}{\bfseries Security} \\ %TODO transpose
                           & \bfseries Konfidenz & \bfseries Verbreitung & \bfseries Konfidenz & \bfseries Verbreitung \\
    \hline \\
    Praktische Tests          & +++    & +++     & +           & ++       \\
    Proven-in-use             & ++     & +       & -           & +        \\
    Mathematische Beweise     & +++    & -       & +++         & ++       \\
    Externe Audits            & ++     & +++     & +++         & +++      \\
    \end{tabular}
    \caption{
      Übersicht der Akzeptanzkriterien in Safety und Security\\
      \textbf{Verbreitung} Häufigkeit als tragendes Argument im Assurance-Case\\
      \textbf{Konfidenz} Vertrauen in das Kriterium
    }
  \end{table}

  % \begin{itemize}
  %   \item Safety
  %   \begin{itemize}
  %     \item Goldstandard: Requirements-getriebenes Testen % üblich
  %     \item Alternative: "Proven in use" % unüblich
  %     \item Alternative: Formale Verifikation % selten
  %   \end{itemize}

  %   \item Security
  %   \begin{itemize}
  %     \item "Proven in use" reicht nicht, aber neuen Verfahren wird dennoch misstraut
  %     \item Testen unzureichend, aufgrund gezielter Angriffe
  %     \item Formale Verifikation ist der Goldweg
  %   \end{itemize}
  % \end{itemize}

\end{frame}

\begin{frame}[T]{Kulturen}
  % GOAL Es gibt Unterschiede, die Begründet sind. Wenn Safety und Security verbunden werden sollen, müssen beide Gründe verstanden werden.
  \begin{itemize}
    \item Safety
    \begin{itemize}
      \item Menschen Sterben bei Versagen
      \item Probleme sind Verstanden und Stabil
      \item[$\Rightarrow$] Konservative Ingenieurskultur
    \end{itemize}

    \item Security
    \begin{itemize}
      \item Versagen erzeugt eher Finanziellen Schaden
      \item Problemtypen sind dynamisch und ändern sich dauernd
      \item[$\Rightarrow$] Progressive Ingenieurskultur
    \end{itemize}

    \item Security und Safety Kombiniert:
    \begin{itemize}
      \item Menschen sterben bei Versagen
      \item Aber die Probleme sind dynamisch, die Zielsetzung in Bewegung
      \item[$\Leftrightarrow$] Konservativ $\lightning$ Progressiv
    \end{itemize}
  \end{itemize}
\end{frame}

\begin{frame}[T]{Safety + Security Benötigt}
  % GOAL Klar machen, dass beide Domänen nicht unversöhnbar sind
  \begin{itemize}
    \item Hohe Zuverlässigkeit
    \item Klarheit über Systemziele
    \item Rigorose Validierungsprozesse (Unabhängiges Review)
    \item Analyse von Softwaresystemen in reeller Hardware
    \item Redundante Systeme
    \item[$\Rightarrow$] \textbf{Kryptoagilität}
  \end{itemize}
\end{frame}
