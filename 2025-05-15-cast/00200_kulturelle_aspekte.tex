\interlude[0]{Safety \& Security \\ \Large \hrule \vspace{1.2em} Kulturelle Aspekte}
\section{Safety \& Security}

% TODO Definitions slide für Safety und Security hinzufügen

\begin{frame}[T]{Safety \& Security: Unterschiede}
% GOAL Klar machen, dass beide Begriff verschieden sind und verschiedene Vorgehen erfordern
\small
  \begin{columns}[t,fullwidth]
   \hfill
    \begin{column}{.45\linewidth}
      \begin{block}{Safety}
      \begin{itemize}
        \item Zufällige Fehler
        \item Im Fehlerfall: Weiterbetrieb ermöglichen!
        \item Stabile Zieldefinition: Physik bleibt gleich
        \item Abgehangene Software -> Stabile Software!
        \item Rigide Validierungsprozesse
      \end{itemize}
      \end{block}
    \end{column}
    \hfill
    \begin{column}{.45\linewidth}
      \begin{block}{Security}
      \begin{itemize}
        \item Intelligente Angreifer
        \item Im Fehlerfall: Lieber das System stoppen
        \item Zieldefinition ist in Bewegung: Angreifer lernen auch dazu
        \item Abgehangene Software -> CVEs bekannt?
        \item Viele Freiheitsgrade in Validierung
      \end{itemize}
      \end{block}
    \end{column}
    \hfill
  \end{columns}
\end{frame}


\begin{frame}[T]{Safety \& Security: Gemeinsamkeiten}
  % GOAL Klar machen, dass beide Begriffe aber auch wichtige Gemeinsamkeiten haben
  \begin{itemize}
    \item Hohes Zuverlässigkeit nötig
    \item Analyse von Softwaresystemen in reeller Hardware
    \item Rigorose Validierungsprozesse
    \item -> Gesetzliche Anforderungen verpflichten zur Mittelmäßigkeit
    \item Dokumentation von Systemzielen ist sehr wertvoll
  \end{itemize}
\end{frame}

%
\begin{frame}[T]{Akzeptanzkriterien}
  % GOAL erklären, wie man in Safety/Security Konfidenz erzeugt, einen guten Job gemacht zu haben
   % TODO make this horizontal?
  \begin{itemize}
    \item Safety
    \begin{itemize}
      \item Requirements-getriebenes Testen % üblich
      \item "Proven in use" % unüblich
      \item Formale Verifikation % selten
    \end{itemize}

    \item Security
    \begin{itemize}
      \item "Proven in use" reicht nicht, aber neuen Verfahren wird dennoch misstraut
      \item Testen unzureichend, aufgrund gezielter Angriffe
      \item Formale Verifikation ist der Goldweg
    \end{itemize}

    \item Gemeinsamkeit
    \begin{itemize}
      \item Redundanz um Ansprüche an Einzelkomponenten zu senken
      \item Unabhängiges Review
    \end{itemize}
  \end{itemize}
\end{frame}

\begin{frame}[T]{Safety- \& Securitykultur}
  \begin{itemize}
    \item Safety
    \begin{itemize}
      \item Menschen Sterben bei Versagen
      \item Probleme sind Verstanden und Stabil
      \item => Konservative Ingenieurskultur
    \end{itemize}

    \item Security
    \begin{itemize}
      \item Versagen erzeugt eher Finanziellen Schaden
      \item Problemtypen sind dynamisch und ändern sich dauernd
      \item => Progressive Ingenieurskultur
    \end{itemize}
  \end{itemize}
\end{frame}
