\interlude[2]{Kryptoagilität als Prozess}

\begin{frame}{Modulare Systemdesigns}
  \begin{columns}[t,fullwidth]
    \begin{column}{.3\linewidth}
      Möglichkeiten:
      \tiny
      \vspace{0.5em}
      \begin{itemize}
        \item Rapider Komponentenaustausch
        \item Vereinfachtes Systemdesign
        \item Validierung einzelner Komponentent
      \end{itemize}
    \end{column}

    \vrule

    \begin{column}{.65\linewidth}
      Herausforderungen:
      \tiny
      \vspace{0.5em}
      \begin{itemize}
        \item Abstraktionen sind häufig unvollständig
        \item Gute Modulgrenzen finden braucht Erfahrung
        \item Viele funktionelle \& nicht-funktionelle Anforderungen
        \item Schlecht gewählte module bringen eine Illustion von Sicherheit
        \item Cargo Culting
      \end{itemize}
    \end{column}

  \end{columns}

  \hrule

  Ein agiler Lebenszyklus für Module

    \tiny
  \begin{itemize}
    \item Laufende, agile Modulentwicklung
    \item Enge zusammenarbeit zwischen Modul- und Systemdesignern
    \item Einsatz besonders Erfahrener Ingeneaure
    \item Breite, industrieübergreifende Zusammenarbeit hilfreich
  \end{itemize}

\end{frame}


\begin{frame}{Korrekt von Anfang an}
  \begin{columns}[t,fullwidth]
    \begin{column}{.3\linewidth}
      Reaktiv vs Proaktiv
      \tiny
      \vspace{0.5em}
      \begin{itemize}
        \item Externe Validierung essenziell (Sicherheitsbeweise, Zertifizierung)
        \item Ansätze können kombiniert werden
        \item Proaktive Methoden sind in Safety und Security weit verbreitet
      \end{itemize}
    \end{column}

    \vrule

    \begin{column}{.65\linewidth}
      Proaktives Vorgehen ist besonders schwer
      \tiny
      \vspace{0.5em}
      \begin{itemize}
        \item Avionik: Zertifizierung ist schwer
        \item Kryptografie: Formelle Beweise sind schwer
        \item Die Bürde der Verifikation erdrückt Innovation
      \end{itemize}
    \end{column}

  \end{columns}

  \hrule

  Schritte für den Anfang

  \tiny
  \begin{itemize}
    \item Investition in Lehrmaterialien, Handbücher, Nutzerfreundliche Verifikationssysteme
    \item Benutzbare Beweistools, Menschenfreundliche Zertifizierungsbehörden
    \item Innovation für bessere, effektivere Verifikation müssen gefördert werden
    \item Kontinuierliche Verifikation in Aktiver Zusammenarbeit zwischen allen Aktören
  \end{itemize}
\end{frame}


\begin{frame}{Konsequenzen \& Chancen}
  \begin{itemize}
    \item  Sicherheit: Hochgradig formalisierte Prozesse
    \item  Krypto: Strenge ethische, weniger formalisierte Prozesse 
    \item  => Shannons Prinzip
    \item  => Verantwortungsvolle Offenlegung
    \item  => Staatlich vorgeschriebene Berichterstattung über Schwachstellen
    \item  => Es geht darum, eine Umgebung zu schaffen, in der man aus Fehlern lernen kann
  \end{itemize}
\end{frame}
