\interlude[3]{Kryptoagilität \\ \hrule \normalsize Eine Definition }

\begin{frame}[light]{}
  Fortschritt benötigt fortschrittliche Prozesse:
\end{frame}


\begin{frame}[light]{}
\begin{itemize}
  \item Abkehr von der technokratischen Perspektive auf Agilität: Es geht um Prozessgestaltung und soziale Realitäten % Formulierungsidee: Kryptoagilität ist keine Library und auch kein Tool was man installiert. Es ist eine Fähigkeit, die durch umfassende Prozesse Stück für Stück erarbeitet wird.
  \item Kontinuierliche Bereitstellung: Solange das Produkt eingesetzt wird, ist die Entwicklung nicht beendet
  \item Fehler akzeptieren – Ein Argument für Bescheidenheit 
  \item     Planen Sie für Fehlerszenarien, um auf Fehlerszenarien einfach und professionell reagieren zu können
  \item     Keine Schuldzuweisungen sondern Ursachenfindung für bestmögliche Reaktionen auf Vorfälle
  \item     Scham und Bestrafung für absichtliches Verschweigen von Problemen
  \item     Aufbau einer Infrastruktur zur Unterstützung von Einsatzkräften – Serviceorientierung statt Schuldzuweisungen % TODO karo: was sind Einsatzkräft? Meint dies Personal für die Reaktion auf Vorfälle? Dann würde ich vielleicht von Incident Response Teams sprechen.
\end{itemize}
\end{frame}
