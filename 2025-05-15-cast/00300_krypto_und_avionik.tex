\interlude[1]{Kryptografie in der Avionik}

% TODO klar benennen, dass wir uns jetzt ein Fallbeispiel Anschauen

\begin{frame}{Die Vier Domänen der Sicherheit sind…}
  % GOAL Orientierung, wo wie gerade hinschauen für das Fallbeispiel
  \begin{columns}[c]
    \begin{column}{.5\linewidth}
      \textbf{Luftfahrt}
    \end{column}
    \begin{column}{.5\linewidth}
      Automobile
    \end{column}
  \end{columns}
  \vfill
  \begin{columns}[c]
    \begin{column}{.5\linewidth}
      Medizintechnik
    \end{column}
    \begin{column}{.5\linewidth}
      Automatisierung
    \end{column}
  \end{columns}
\end{frame}


\begin{frame}[c]{Zum erschrecken aller…}
  % TODO: Screenshot von Economy Class cryptography Paper, und von LDACS paper
  % TODO: Both papers in two-column?
  % GOAL Aufzeigen, das gut Cryptography in Avionik keine Selbstverständlichkeit ist
  % GOAL Aufzeigen, dass auch fortschrittliche Planung durch Mangelnde Kryptoagilität zum scheitern verurteilt ist (LDACS Paper)
  \vspace{15em}
  \footnotesize
  …wird in der Luftfahrt heutzutage keine sichere Kryptografie eingesetzt. 
\end{frame}

\begin{frame}[c]{Zum erschrecken aller…}
  \vspace{10em}
  \footnotesize
  …wird in der Luftfahrt heutzutage keine sichere Kryptografie eingesetzt. 

  \vspace{1em}
  Der eine Ernstzunehmende Vorschlag den wir gefunden hatten -- ein System zum kryptografischen absichern von LDACS --
  setzte auf post-quanten chiffren (SIKE), gegen die ein Jahr später Angriffe gefunden wurden.

  \vspace{1em}
  TODO: Insert REFERENCE
\end{frame}

% TODO: neue Folie zu unserem Paper
% GOAL Aufzeigen, dass und wie Modularität der Schlüssel für Kryptoagilität in der Safety ist
% GOAL Aufzeigen, dass Integration einfach möglich, und zulassungsfreundlich gemacht werden kann

\begin{frame}[c]{Ansätze für Verbesserung in beiden Bereichen}
  % GOAL ist mir unklar
  % TODO wucke refine 

  Verifikationstechniken
  \vspace{0.5em}

  {\footnotesize Zertifizierung <-> Sicherheitsbeweise}
  \vspace{1.5em}
  % Formaler Beweis kann Teil der Evidence für Zertifizierung sein
  % Annahmen und Anforderungen müssen in den Zertifizierungsdokumenten festgehalten werden
  % Evidence = Belege, warum Annahmen wahr sind oder Anforderungen erfüllt sind
  % Unabhängigkeit zwischen Implementierer und Verifizierer
  % 
  % - Wanja
  %
  % Agreed; geht mir eher darum die generelle Vorgehensweise zu beschreiben als formell korrekt jede Möglichkeit durchzugehen.
  % Ist beweisführung in der Avionik denn üblich?
  % 
  % – Karolin

  Kompartmentalisierung
  \vspace{0.5em}

  {\footnotesize Partitionierung <-> Brokerarchitekturen}
  \vspace{1.5em}
  % Aufteilung erleichtert sukzessive Absicherung einzelner Komponent
  % Gleichzeitig: Faultcontainment
  % Modularisierung macht updates einfacher, da klare interfaces zwischen Modulen

  % Diagram Idee: Konfidenzarchitektur
  %
  % 1. Identifiziere mögliche Fehlerursachen
  % 2. Ist ein Fehler relevant?
  %   - Falls Nein, warum nicht? Evidence!
  % 3. Welche Effekte hat der Fehler?
  % 4. Wie kann der Fehler verhindert/eingedämmt/mitigiert werden?
  % 5. Wie kann der Fehler erkannt werden?
  % 6. Validiere Annahmen von 3. - 5.
  % 7. Review durch unabhängigen Assesor
  %
  % -- wucke13
  %
  % Passt das denn in das Gegenüberstellungsscheme? Also wie bringen wir den Vergleich Avionik vs Crypto rein.
  %
  % -- karolin

  => Kryptoagilität
\end{frame}
