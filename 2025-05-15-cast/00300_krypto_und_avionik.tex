\interlude[1]{Kryptografie und Avionik im Dialog}

\begin{frame}{Die Vier Domänen der Sicherheit sind…}
  \begin{columns}[c]
    \begin{column}{.5\linewidth}
      \textbf{Luftfahrt}
    \end{column}
    \begin{column}{.5\linewidth}
      Automobile
    \end{column}
  \end{columns}
  \vfill
  \begin{columns}[c]
    \begin{column}{.5\linewidth}
      Medizintechnik
    \end{column}
    \begin{column}{.5\linewidth}
      Automatisierung
    \end{column}
  \end{columns}
\end{frame}


\begin{frame}[c]{Zum erschrecken aller…}
  \vspace{15em}
  \footnotesize
  …wird in der Luftfahrt heutzutage keine sichere Kryptografie eingesetzt. 
\end{frame}

\begin{frame}[c]{Zum erschrecken aller…}
  \vspace{10em}
  \footnotesize
  …wird in der Luftfahrt heutzutage keine sichere Kryptografie eingesetzt. 

  \vspace{1em}
  Der eine Ernstzunehmende Vorschlag den wir gefunden hatten -- ein System zum kryptografischen absichern von LDACS --
  setzte auf post-quanten chiffren (SIKE), gegen die ein Jahr später Angriffe gefunden wurden.

  \vspace{1em}
  TODO: Insert REFERENCE
\end{frame}

\begin{frame}[c]{Ansätze für Verbesserung in beiden Bereichen}

  Verifikationstechniken
  \vspace{0.5em}

  {\footnotesize Zertifizierung <-> Sicherheitsbeweise}
  \vspace{1.5em}

  Kompartmentalisierung
  \vspace{0.5em}

  {\footnotesize Partitionierung <-> Brokerarchitekturen}
  \vspace{1.5em}

  => Kryptoagilität
\end{frame}
