\interlude[2]{Erkenntnisse \\ \Large \hrule \vspace{1.2em} Kryptoagilität als Prozess}

\begin{frame}{Modulare Systemdesigns}
  % GOAL aufzeigen, dass Modularität schlüssel ist
  % TODO das ist eine wiederholung zu unserem Paper, vielleicht entfernen?
  \begin{columns}[t,fullwidth]
    \begin{column}{.3\linewidth}
      Möglichkeiten:
      \tiny
      \vspace{0.5em}
      \begin{itemize}
        \item Rapider Komponentenaustausch
        \item Vereinfachtes Systemdesign
        \item Separate Validierung einzelner Komponentent
      \end{itemize}
    \end{column}

    \vrule

    \begin{column}{.65\linewidth}
      Herausforderungen:
      \tiny
      \vspace{0.5em}
      \begin{itemize}
        \item Abstraktionen sind häufig unvollständig
        \item Gute Modulgrenzen finden braucht Erfahrung
        \item Viele funktionelle \& nicht-funktionelle Anforderungen
        \item Schlecht gewählte module bringen eine Illustion von Sicherheit
        \item Cargo Culting
      \end{itemize}
    \end{column}

  \end{columns}

  \hrule

  Ein agiler Lebenszyklus für Module

    \tiny
  \begin{itemize}
    \item Laufende, agile Modulentwicklung
    \item Enge zusammenarbeit zwischen Modul- und Systemdesignern
    \item Einsatz besonders Erfahrener Ingenieure
    \item Breite, industrieübergreifende Zusammenarbeit hilfreich
    \item Continuous Certification Process
    \item Continuous Deployment Infrastruktur
  \end{itemize}

\end{frame}


\begin{frame}{Korrekt von Anfang an}
  % GOAL Zeigen, das Proaktives Vorgehen sich auszeichnet, und sogar notwendig ist um Reaktives Verhalten (=> Kryptoagilität) zu ermöglichen
  \begin{columns}[t,fullwidth]
    \begin{column}{.3\linewidth}
      Reaktiv vs Proaktiv
      \tiny
      \vspace{0.5em}
      \begin{itemize}
        \item Externe Validierung essenziell (Sicherheitsbeweise, Zertifizierung)
        \item Ansätze sollten kombiniert werden; "Defense in Depth"
        \item Proaktive Methoden sind in Safety und Security weit verbreitet
      \end{itemize}
    \end{column}

    \vrule

    \begin{column}{.65\linewidth}
      Proaktives Vorgehen ist besonders schwer
      \tiny
      \vspace{0.5em}
      \begin{itemize}
        \item Hoher Grad an präziser Planung erforderlich % Benötigt Turbo-Glaskugel
        \item Avionik: Zertifizierung ist schwer
        \item Kryptografie: Formelle Beweise sind schwer
        \item Die Bürde der Verifikation erdrückt Innovation
      \end{itemize}
    \end{column}

  \end{columns}

  \hrule

  Schritte für den Anfang

  \tiny
  \begin{itemize}
    \item Investition in Lehrmaterialien, Handbücher, Nutzerfreundliche Verifikationssysteme
    \item Benutzbare Beweistools, Menschenfreundliche Zertifizierungsbehörden
    \item Innovation für bessere, effektivere Verifikation müssen gefördert werden
    \item Erhöhung der Bringschuld nur Hand in Hand mit Ausbau von Tooling und Prozesshilfen
    \item Kontinuierliche Verifikation in Aktiver Zusammenarbeit zwischen allen Akteuren
  \end{itemize}
\end{frame}


\begin{frame}{Konsequenzen \& Chancen}
  % GOAL unklar
  % TODO klar machen, das staatliche Regulireung mit staatlicher Investition in Prozesse und Tooling hand in hand gehen muss -> Wettbewerbsvorteil falls ja, wettbewerbsnachteil falls nein
  % TODO klar machen, dass wir von Safety Prozessen lernenen können, insbesodere was dokumentation von anforderungen angeht -> Ohne dokumentierte Anforderungen ist Kryptoagilität schwierig.
  % TODO Gefahr bennen, das staatliche über-regulierung zur Mittelmäßigkeit verpflichtet
  \begin{itemize}
    \item  Sicherheit: Hochgradig formalisierte Prozesse
    \item  Krypto: Strenge ethische, weniger formalisierte Prozesse 
    \item  => Kerckhoff Prinzip
    \item  => Verantwortungsvolle Offenlegung
    \item  => Staatlich vorgeschriebene Berichterstattung über Schwachstellen
    \item  => Es geht darum, eine Umgebung zu schaffen, in der man aus Fehlern lernen kann
  \end{itemize}
    % TODO this is a challenge, not a solution -> move somewhere
    \item -> Gesetzliche Anforderungen sind Chance und Gefahr
    \begin{itemize}
      \item verpflichten zur Mittelmäßigkeit
    \end{itemize}
\end{frame}
