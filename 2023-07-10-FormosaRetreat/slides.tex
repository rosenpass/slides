\documentclass{rosenpass-beamer}

\usepackage[ngerman]{babel}

\usepackage{textcomp}

\usetikzlibrary{positioning,decorations.pathreplacing,svg.path}

\newcommand*{\heading}[1]{
  {
    \hspace*{-0.5cm}#1
    \vspace{1.0em}
  }
}

\title{Rosenpass}
%\subtitle{VPN \& Struktur für Translationsforschung in der Kryptografie}
\author{
Wanja Zaeske, Stephan Ajuvo, Marei Peischl, Benjamin Lipp, Lisa~Schmidt, Karolin Varner
}
\institute{\url{https://rosenpass.eu}}


\conference{Formosa Retreat Juli 2023}
\date{2023-07-11}

\parskip\smallskipamount

% reduce itemize indent
\setlength{\leftmargini}{0pt}

\usepackage{biblatex}
\addbibresource{sources.bib}


\graphicspath{{}{graphics/}}

\begin{document}

\maketitle

\begin{frame}{Hello, I am Karolin Varner}
\begin{itemize}
  \item Worked with about every industry tech; incl. Java Web Apps, Microcontrollers, and legacy database system from the 80s
  \item Did a lot of project management and some people management
  \item Did a lot of open-source development, privacy- and internet politics activism
  \item Planning to get involved in the Formosa space
\end{itemize}
\end{frame}

\begin{frame}{Rosenpass}

\vspace{0.5em}
\begin{columns}[t]
\begin{column}{.30\textwidth}
\heading{WireGuard}
\begin{itemize}
  \itemtick Session-key secrecy
  \itemtick \dots
  \itemtick Identity Hiding
  \itemfail \textbf{Non-Interruptability} \footnote[frame]{Assuming a trusted system time}
  \itemfail \textbf{Post-Quantum Security}
\end{itemize}
\end{column}

\begin{column}{.30\textwidth}
\heading{
  PQ WireGuard
  \footnote[frame]{
	  Hülsing, Ning, Schwabe, Weber, Zimmermann. “Post-quantum WireGuard”. https://ia.cr/2020/379
	}
}
\begin{itemize}
  \itemtick \textbf{Post-Quantum Security}
  \itemfail \textbf{Hybrid security}
  \itemfail \textbf{Non-Interruptability} \footnote[frame]{Assuming a PSK}
\end{itemize}
\end{column}

\begin{column}{.30\textwidth}
\heading{Rosenpass}
\begin{itemize}
  \itemtick \textbf{Non-Interruptability} \footnote[frame]{Through cookies}
  \itemtick \textbf{Hybrid security} \footnote[frame]{Used together with standard WireGuard}
\end{itemize}
\end{column}

\end{columns}
\vspace{1.5em}

\end{frame}

%\begin{frame}{Rosenpass/WireGuard integration}
\input{rpwg-graphic}
%\end{frame}

\begin{frame}{Rosenpass can be used right now}
  \includegraphics[height=.9\textheight]{assets/2023-03-20-rg-tutorial-screenshot.png}
\end{frame}

\begin{frame}{ProVerif in Technicolor}
  \includegraphics[height=.9\textheight]{assets/2023-03-20-symbolic-analysis-screenshot.png}
\end{frame}

\begin{frame}{Having worked in industry has some advantages}
\begin{itemize}
  \item Knowing how to get projects done
  \item Coordinating teams instead of working on my own
  \item Product and user focused perspective
  \item Building tools that I can use to be more productive
  \item Open-Source approach: How to catch new contributors
\end{itemize}
\end{frame}

\begin{frame}{The spec makes it easy to implement Rosenpass}
  \includegraphics[height=.9\textheight]{rosenpass-whitepaper-message-handling-code.pdf}
\end{frame}

\begin{frame}{Professional illustrators create stunning graphics}
  \includegraphics[height=.9\textheight]{rosenpass-whitepaper-hashing-tree.pdf}
\end{frame}

\begin{frame}{Creating successful projects by knowing what not to do}
\begin{itemize}
  \item Rosenpass avoids targeting: GUIs, VPN data transport, support for many platforms
  \item Instead we: Created a core technology; working with companies to integrate Rosenpass (e.g. Open-Source VPN startups)
  \item \textbf{Vitally we} chose to focus on API; making it easy to integrate Rosenpass
  \item \textbf{Vitally we} integrate with the existing ecosystem (i.e. WireGuard) instead of trying to replace it
\end{itemize}
\end{frame}

\begin{frame}{Starting partnerships\dots}
\begin{itemize}
  \item with Open-Source VPN companies
  \item with Kubernetes VPN companies
  \item with Quantum-Key-Distribution Projects
  \item to verify the Rosenpass source code
  \item to apply isolation features to Rosenpass (Micro-VMs)
  \item with university teaching departments to use the project as a simple example of bleeding-edge modern crypto
\end{itemize}
\end{frame}

\begin{frame}{Talk to me about\dots}
\vspace{0.5cm}
\begin{itemize}
  \item using Rosenpass as demonstrator-project to integrate new cryptographic technologies in
  \item figuring out how to attract independent contributors to Formosa Projects
  \item applying API-focused techniques to Formosa projects to emphasize interoperability
  \item \textbf{Idea:} Providing XML-representations of proof assistants'\footnote{EasyCrypt, CryptoVerif, ProVerif, Tamarin} inputs and outputs to allow easy integration with external tools
  \item \textbf{Idea:} Python libraries to work with Formosa tools as everybody knows python
\end{itemize}
\vspace*{1.5cm}
\end{frame}

\end{document}

\printbibliography

\clearpage
\setcounter{framenumber}{\totalcontentframes}



\end{document}
