\interlude{Was ist Kryptografie}
\section{Was ist Kryptografie}


\begin{frame}[T]{Was ist Kryptografie}

% Absicherung von Kommunikation.

% Im Internet, aber auch woanders.


\begin{columns}[t,fullwidth]
  \hfill
  \begin{column}{.45\linewidth}
    Bild: Schutz Privater Kommunikation "Niemand soll es Mitlesen"
    \begin{itemize}
      \item Küchentisch
      \item Briefgeheimnis
      \item Arztgespräch
    \end{itemize}
  \end{column}
  \begin{column}{.45\linewidth}
    Bild Schutz vor Kriminalität
    \begin{itemize}
      \item Bankraub (Online Banking)
      \item Betrug
      \item Spionage
      \item Vandalismus
    \end{itemize}
  \end{column}
  \hfill
\end{columns}
Räume im Internet, die wie im Rest des Lebens funktionieren.

% -> Viel heutzutage im Internet
% -> Die Kommunikation muss abgesichert werden
\end{frame}

\begin{frame}[T]{Das Internet ist öffentlich}
  Bild: 
  \begin{itemize}
    \item Öffentlicher Platz mit ganz viel geheimer Kommunikation die öffentlich geteilt wird
  \end{itemize}
\end{frame}

\begin{frame}[T]{Wie das funktioniert}
  Bild: 
  \begin{itemize}
    \item Patient und Doktor sprechen via Internet
    \item Zwei computer (oder andere geräte) kommunizieren miteinander
    \item Haben geheime Schlüssel
    \item Absicherung von Kommunikation via Mathematik
    \item Internet ist an sich öffentlich
  \end{itemize}
\end{frame}


%\begin{frame}[T]{Kryptografische Sicherheit ist exponentiell}
%  \begin{itemize}
%    \item Eine Einheit mehr Arbeit für ehrliche Parteien
%    \item => Doppelte Arbeit für Angreifer (oder ein anderer Faktor)
%  \end{itemize}
%\end{frame}
%
%\begin{frame}[T]{Exponentialrechnung: Krishna's Rice Pudding}
%  Bild: 
%  \begin{itemize}
%    \item Ambalappuzha Sree Krishna Swamy Temple Legend
%    \item \url{https://en.wikipedia.org/wiki/Ambalappuzha_Sree_Krishna_Swamy_Temple}
%    \item Schachbrett mit figuren
%    \item "Hey, können sie die Weltreisproduktion um den Faktor 1300 erhöhen"
%    \item Höchster turm reicht bis weit hinter pluto (13 licht-tage)
%    \item Planet pluto mit messer und gabel
%  \end{itemize}
%\end{frame}
%
%\begin{frame}[T]{Exponentialrechnung: Weltwirtschaftswachstum}
%  \begin{itemize}
%    \item Angenommen Wirtschaftswachstum von 2.5\%
%    \item Angenommen Energienutzung proportional zum Wirtschaftswachstum
%    \item Erdenergiebudget in 300 Jahren überschritten
%    \item Weitere 600 Jahre: Sonnenenergiebudget überschritten
%    \item Weitere 2500 Jahre: Energiebudget des Universums überschritten
%    \item (Das geht nicht weil Lichtgeschwindigkeit)
%  \end{itemize}
%\end{frame}
%
%\begin{frame}[T]{Exponenzielle Sicherheit}
%  \begin{itemize}
%    \item Kleinstmögliche energiekosten (Landauer Limit) für kostenschritte
%    \item Energiebudget der Erde verbraucht: ~150 Verdopplungsschritte oder 4200 schritte bei faktor 2.5\%
%    \item Energiebudget der Sonne verbraucht: ~180 Verdopplungsschritte oder 5000 bei faktor 2.5\%
%    \item Energieinhalt des Universums verbraucht: ~300 Verdopplungsschritte oder 8500 bei faktor 2.5\%
%  \end{itemize}
%\end{frame}
