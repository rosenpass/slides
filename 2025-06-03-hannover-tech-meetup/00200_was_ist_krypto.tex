\interlude{Was ist Kryptografie}
\section{Was ist Kryptografie}


\begin{frame}[T]{Sichere Kommunikationsräume Schaffen}
  \begin{columns}[t,fullwidth]
    \hfill
    \begin{column}{.35\linewidth}
      \centering
      \includegraphics[width=.305\pagewidth]{comic/rosenpass-comic-diary.png}   
      \\ Schutz von Privatem
    \end{column}
    \begin{column}{.55\linewidth}
      \centering
      \includegraphics[width=.30\pagewidth]{comic/rosenpass-comic-safe.png}   
      \\ Schutz vor Diebstahl \& Vandalismus
    \end{column}
    \hfill
  \end{columns}
% -> Viel heutzutage im Internet
% -> Die Kommunikation muss abgesichert werden
\end{frame}

\begin{frame}[T]{Digitale Räume, so sicher wie die analogen}
  \includegraphics[height=\defaultframetextheight]{comic/rosenpass-comic-rgm-1.png}
\end{frame}

\begin{frame}[T]{Datenkommunikation ist öffentlich}
  \includegraphics[height=\defaultframetextheight]{comic/rosenpass-comic-rgm-2.png}
\end{frame}

\begin{frame}[T]{Datenströme wie Kauderwelsch}
  \begin{columns}[T,fullwidth]
    \hfill
    \begin{column}{.55\linewidth}
      \includegraphics[height=\defaultframetextheight,trim=65 0 65 0,clip]{comic/rosenpass-comic-rgm-2.png}
    \end{column}
    \begin{column}{.38\linewidth}
      \vspace{2.3em}
      \begin{itemize}
        \item Patient und Doktor tauschen geheime Zahlen aus
        \item Beide Computer verschlüsseln
        \item Patient und Doktor verstehen sich
        \item Für alle anderen ist der Datenstrom Kauderwelsch
        \item Ordentlich umgesetzt sehen sie nicht mal, wer mit wem spricht
      \end{itemize}
    \end{column}
    \hfill
  \end{columns}
\end{frame}


%\begin{frame}[T]{Kryptografische Sicherheit ist exponentiell}
%  \begin{itemize}
%    \item Eine Einheit mehr Arbeit für ehrliche Parteien
%    \item => Doppelte Arbeit für Angreifer (oder ein anderer Faktor)
%  \end{itemize}
%\end{frame}
%
%\begin{frame}[T]{Exponentialrechnung: Krishna's Rice Pudding}
%  Bild: 
%  \begin{itemize}
%    \item Ambalappuzha Sree Krishna Swamy Temple Legend
%    \item \url{https://en.wikipedia.org/wiki/Ambalappuzha_Sree_Krishna_Swamy_Temple}
%    \item Schachbrett mit figuren
%    \item "Hey, können sie die Weltreisproduktion um den Faktor 1300 erhöhen"
%    \item Höchster turm reicht bis weit hinter pluto (13 licht-tage)
%    \item Planet pluto mit messer und gabel
%  \end{itemize}
%\end{frame}
%
%\begin{frame}[T]{Exponentialrechnung: Weltwirtschaftswachstum}
%  \begin{itemize}
%    \item Angenommen Wirtschaftswachstum von 2.5\%
%    \item Angenommen Energienutzung proportional zum Wirtschaftswachstum
%    \item Erdenergiebudget in 300 Jahren überschritten
%    \item Weitere 600 Jahre: Sonnenenergiebudget überschritten
%    \item Weitere 2500 Jahre: Energiebudget des Universums überschritten
%    \item (Das geht nicht weil Lichtgeschwindigkeit)
%  \end{itemize}
%\end{frame}
%
%\begin{frame}[T]{Exponenzielle Sicherheit}
%  \begin{itemize}
%    \item Kleinstmögliche energiekosten (Landauer Limit) für kostenschritte
%    \item Energiebudget der Erde verbraucht: ~150 Verdopplungsschritte oder 4200 schritte bei faktor 2.5\%
%    \item Energiebudget der Sonne verbraucht: ~180 Verdopplungsschritte oder 5000 bei faktor 2.5\%
%    \item Energieinhalt des Universums verbraucht: ~300 Verdopplungsschritte oder 8500 bei faktor 2.5\%
%  \end{itemize}
%\end{frame}
