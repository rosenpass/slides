\interlude{Migration zur Post-Quanten Sicherheit}

\NewDocumentCommand{\timelineentry}{O{#2}mms}{
	\node[anchor=north west] (#2) at ([yshift=-1.15ex]lastnode)  {#1: #3};
	\coordinate(lastnode) at (#2.south west);
	% Frame?
%	\draw(#2.north west)--(#2.south west) -- ++(1cm,0);
	\IfBooleanF{#4}{
		\draw(0,#2)--(#2.west);
		\fill[](0,#2) circle [radius=.2ex];
	}
}

\begin{frame}[T]{Migration zur Post-Quantum-Sicherheit}
\begin{tikzpicture}[y=-0.013\textheight,rosenpass-diagram]
	\draw[->,rosenpass-blue]
	(0,1973)--(0,1973.5)
	(0,1974)--(0,1974.3)
	(0,1974.6)--(0,1975)
	(0,1975.5)--(0,2030);
	\foreach \year in {1980,1990,...,2020} {
	\draw (-.1,\year) node[left,font=\footnotesize,rosenpass-gray]{\year}--++(.2,0);
	}
	\coordinate (lastnode) at (.2\textwidth,1970);
	\timelineentry{1978}{Classic McEliece: 1. PQ-Sichere Chiffre}
	\timelineentry{1994}{Kritische Quantenangriffe gefunden}
	\timelineentry{2016}{Wettbewerb angekündigt: PQ-Sichere Standardchiffren}% – https://csrc.nist.gov/Projects/post-quantum-cryptography/news
	\timelineentry{2019}{Experiment: PQ-Sicherheit auf öffentlichen Websiten}% – https://blog.cloudflare.com/towards-post-quantum-cryptography-in-tls/
	\timelineentry{2022}{OpenSSH abgesichert}% – https://www.openssh.com/txt/release-9.0
	\timelineentry{2023}{Rosenpass veröffentlicht (WireGuard abgesichert)}
	\timelineentry{2023}{Signal Messenger abgesichert }% began migration – https://signal.org/blog/pqxdh/
	\timelineentry[seit 2024]{2024}{NIST-Wettbewerb: Erster Standard verabschiedet }*% – https://csrc.nist.gov/News/2024/postquantum-cryptography-fips-approved
	\timelineentry[Zukunft]{2030}{Umfassender einsatz von PQ-Sicherheit}*% –
	\draw[->](2024.west)to[out=220,in=80](-2mm,2028);
	\draw[->](2030.west)to[out=190,in=70](2mm,2031);
	\end{tikzpicture}
\end{frame}

\begin{frame}[T]{Die Systeme sind die Probleme}

Bild:
\begin{itemize}
  \item Doktor / Patient kommunizieren
  \item Dritter Server der Schlüssel verteilt (Vierter fünter server)
  \item Pfeil auf patientencomputer: Windows XP, Virusverseucht
  \item Mensch mit Besen beim Arzt "Haut computer wenn das internet stottert"
  \item "Heriberts-Kneipe" – Promo USB Stick (Einziger Speicher der Geheimen Schlüssel) steckt im Zertifikatscomputer
\end{itemize}

Sichere Verschlüsselungssysteme bestehen aus vielen Komponenten,
die müssen alle abgesichert werden.
\end{frame}

% \begin{frame}[T]{Und inkompatible Bausteine auch}

% Bild:
% \begin{itemize}
%   \item NIST: "Hey we have this shiny new encryption method from the nist competition" (Visualize as puzzle piece)
%   \item Protocol designer: "It does not work in my system"
%   \item Outcome one: "Systemupgrade als ganzes"
%     \begin{itemize}
%         \item Sweaty engineer: I made it fit
%     \end{itemize}
%   \item Outcome two: "Systemupgrade mit Zusatzkomponente"
%     \begin{itemize}
%       \item I bolted an extra heisenberg crypto condensator
%     \end{itemize}
% \end{itemize}

% \end{frame}

\begin{frame}[T]{Kryptoagilität}

\begin{itemize}
  \item Es gibt keine Garantie dass Kryptografische Systeme für immer sicher bleiben
  \item Wir müssen bei der Aktuellen Migration systeme so umbauen, dass zukünftige migration einfacher wird
\end{itemize}

Bild: Buzzer "Crypto agility" mit Hand die ihn Drückt

\end{frame}
