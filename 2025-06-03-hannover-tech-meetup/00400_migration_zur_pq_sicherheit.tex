\interlude{Migration zur Post-Quanten Sicherheit}

\begin{frame}[T]{Migration zur Post-Quantum-Sicherheit}

Bild: Timeline
\begin{itemize}
  \item 1978: McEliece Kryptosystem publiziert (erstes mit PQ-Sicherheit) 
  \item 1994: Shors Algorithmus Publiziert (Quantenangriffe Entdeckt) %  – https://en.wikipedia.org/wiki/Shor%27s_algorithm
  \item 2016: NIST-Wettbewerb für PQ-Sichere Kryptosysteme angekündigt % – https://csrc.nist.gov/Projects/post-quantum-cryptography/news
  \item 2019: Experiment zur nutzung von PQ-Sicherheit auf Websiten % – https://blog.cloudflare.com/towards-post-quantum-cryptography-in-tls/
  \item 2022: OpenSSH-Release abgesichert % – https://www.openssh.com/txt/release-9.0
  \item 2023: Rosenpass veröffentlicht (WireGuard abgesichert)
  \item 2023: Signal Messenger abgesichert % began migration – https://signal.org/blog/pqxdh/
  \item 2024: NIST-Wettbewerb führt zum ersten Standard % – https://csrc.nist.gov/News/2024/postquantum-cryptography-fips-approved
  \item Zukunft: Umfassender einsatz von PQ-Sicherheit
\end{itemize}

\end{frame}

\begin{frame}[T]{Die Systeme sind die Probleme}

Bild:
\begin{itemize}
  \item Doktor / Patient kommunizieren
  \item Dritter Server der Schlüssel verteilt (Vierter fünter server)
  \item Pfeil auf patientencomputer: Windows XP, Virusverseucht
  \item Mensch mit Besen beim Arzt "Haut computer wenn das internet stottert"
  \item "Heriberts-Kneipe" – Promo USB Stick (Einziger Speicher der Geheimen Schlüssel) steckt im Zertifikatscomputer
\end{itemize}

Sichere Verschlüsselungssysteme bestehen aus vielen Komponenten,
die müssen alle abgesichert werden.
\end{frame}

% \begin{frame}[T]{Und inkompatible Bausteine auch}

% Bild:
% \begin{itemize}
%   \item NIST: "Hey we have this shiny new encryption method from the nist competition" (Visualize as puzzle piece)
%   \item Protocol designer: "It does not work in my system"
%   \item Outcome one: "Systemupgrade als ganzes"
%     \begin{itemize}
%         \item Sweaty engineer: I made it fit
%     \end{itemize}
%   \item Outcome two: "Systemupgrade mit Zusatzkomponente"
%     \begin{itemize}
%       \item I bolted an extra heisenberg crypto condensator
%     \end{itemize}
% \end{itemize}

% \end{frame}

\begin{frame}[T]{Kryptoagilität}

\begin{itemize}
  \item Es gibt keine Garantie dass Kryptografische Systeme für immer sicher bleiben
  \item Wir müssen bei der Aktuellen Migration systeme so umbauen, dass zukünftige migration einfacher wird
\end{itemize}

Bild: Buzzer "Crypto agility" mit Hand die ihn Drückt

\end{frame}
