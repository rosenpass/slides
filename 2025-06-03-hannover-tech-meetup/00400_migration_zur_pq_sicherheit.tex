\interlude{Migration zur Post-Quanten Sicherheit}


\NewDocumentCommand{\timelineentry}{O{#2}mms}{
	\node[anchor=north west] (#2) at ([yshift=-1.15ex]lastnode)  {#1: #3};
	\coordinate(lastnode) at (#2.south west);
	% Frame?
%	\draw(#2.north west)--(#2.south west) -- ++(1cm,0);
	\fill[](0,#2) circle [radius=.2ex];
	\IfBooleanF{#4}{
		\draw(0,#2)--(#2.west);
	}
}

\begin{frame}[T]{Migration zur Post-Quantum-Sicherheit}
\vspace{0.8em}
\begin{tikzpicture}[y=-0.013\textheight,rosenpass-diagram]
	\draw[->,rosenpass-blue]
	(0,1973)--(0,1973.5)
	(0,1974)--(0,1974.3)
	(0,1974.6)--(0,1975)
	(0,1975.5)--(0,2030);
	\foreach \year in {1980,1990,...,2020} {
	\draw (-.1,\year) node[left,font=\footnotesize,rosenpass-gray]{\year}--++(.2,0);
	}
	\coordinate (lastnode) at (.2\textwidth,1970);
	\timelineentry{1978}{Classic McEliece: 1. PQ-Sichere Chiffre}
	\timelineentry{1994}{Kritische Quantenangriffe gefunden}
	\timelineentry{2016}{Wettbewerb angekündigt: PQ-Sichere Standardchiffren}% – https://csrc.nist.gov/Projects/post-quantum-cryptography/news
	\timelineentry{2019}{Experiment: PQ-Sicherheit auf öffentlichen Websiten}% – https://blog.cloudflare.com/towards-post-quantum-cryptography-in-tls/
	\timelineentry{2022}{OpenSSH abgesichert}% – https://www.openssh.com/txt/release-9.0
	\timelineentry{2023}{Rosenpass veröffentlicht (WireGuard abgesichert)}
	\timelineentry{2023}{Signal Messenger abgesichert }% began migration – https://signal.org/blog/pqxdh/
	\timelineentry[seit 2024]{2024}{NIST-Wettbewerb: Erster Standard verabschiedet }*% – https://csrc.nist.gov/News/2024/postquantum-cryptography-fips-approved
	\draw[sharp corners,rosenpass-pink,->,opacity=.5](2024.west)--(0,2024)--++(0,4);
	\draw[draw=rosenpass-pink](2024.west)--(0,2024);
	\fill(0,2024) circle [radius=.2ex];
	\begin{scope}[overlay]
	\timelineentry[Zukunft]{2032}{Umfassender einsatz von PQ-Sicherheit}% –
	\end{scope}
	\end{tikzpicture}
\end{frame}

\begin{frame}[T]{Systemkomplexität ist eine Herausforderung}
  \only<1>{
    \includegraphics[height=\defaultframetextheight]{graphics-repo/comic/rosenpass-comic-rgm-2.png}
  }
  \only<2>{
    \begin{columns}[T,fullwidth]
      \hfill
      \begin{column}{.33\linewidth}
        \vspace{3.8em}
        \begin{itemize}
          \item Unsere Kryptografische Infrastruktur besteht aus vielen Einzelkomponenten
          \item Wir wissen nicht genau welche
          \item Fast alle müssen Migriert werden
        \end{itemize}
      \end{column}
      \begin{column}{.62\linewidth}
        \includegraphics[height=\defaultframetextheight,trim=75 0 0 0,clip]{graphics-repo/comic/rosenpass-comic-rgm-3.png}
      \end{column}
      \hfill
    \end{columns}
  }
% Sichere Verschlüsselungssysteme bestehen aus vielen Komponenten,
% die müssen alle abgesichert werden.
\end{frame}

% \begin{frame}[T]{Und inkompatible Bausteine auch}

% Bild:
% \begin{itemize}
%   \item NIST: "Hey we have this shiny new encryption method from the nist competition" (Visualize as puzzle piece)
%   \item Protocol designer: "It does not work in my system"
%   \item Outcome one: "Systemupgrade als ganzes"
%     \begin{itemize}
%         \item Sweaty engineer: I made it fit
%     \end{itemize}
%   \item Outcome two: "Systemupgrade mit Zusatzkomponente"
%     \begin{itemize}
%       \item I bolted an extra heisenberg crypto condensator
%     \end{itemize}
% \end{itemize}

% \end{frame}

\begin{frame}[T]{Kryptoagilität}
  \begin{columns}[T,fullwidth]
    \hfill
    \begin{column}{.30\linewidth}
      \includegraphics[height=1.2\defaultframetextheight,trim=0 0 385 80,clip]{graphics-repo/comic/rosenpass-comic-rgm-3.png}
    \end{column}
    \begin{column}{.65\linewidth}
      \vspace{5em}
      \begin{itemize}
        \item Dokumentieren welche Kryptografische Infrastruktur vorliegt
        \vspace{1.2em}
        \item Prozesse für einen schnellen Austausch etablieren
        \vspace{1.2em}
        \item Nachhaltig und Dauerhaft
      \end{itemize}
    \end{column}
    \hfill
  \end{columns}
\end{frame}
