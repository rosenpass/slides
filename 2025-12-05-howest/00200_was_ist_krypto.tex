\interlude{What is Cryptography?}
\section{What is Cryptography?}

\begin{frame}[T]{Creating trustful communication spaces}
  \begin{columns}[t,fullwidth]
    \hfill
    \begin{column}{.35\linewidth}
      \centering
      \includegraphics[width=.305\pagewidth]{comic/rosenpass-comic-diary.png}   
      \\ Protecting privacy
    \end{column}
    \begin{column}{.55\linewidth}
      \centering
      \includegraphics[width=.30\pagewidth]{comic/rosenpass-comic-safe.png}   
      \\ Protecting belongings \& capital
    \end{column}
    \hfill
  \end{columns}
% -> Viel heutzutage im Internet
% -> Die Kommunikation muss abgesichert werden
\end{frame}

\begin{frame}[T]{Digital spaces, as trustful as the analog ones}
  \includegraphics[height=\defaultframetextheight]{comic/rosenpass-comic-rgm-1.png}
\end{frame}

\begin{frame}[T]{Data communication is usually public}
  \includegraphics[height=\defaultframetextheight]{comic/rosenpass-comic-rgm-2.png}
\end{frame}

\begin{frame}[T]{Data streams become gibberish}
  \begin{columns}[T,fullwidth]
    \hfill
    \begin{column}{.55\linewidth}
      \includegraphics[height=\defaultframetextheight,trim=65 0 65 0,clip]{comic/rosenpass-comic-rgm-2.png}
    \end{column}
    \begin{column}{.38\linewidth}
      \vspace{0.8em}
      \begin{itemize}
        \item Patient and doctor exchange a secret number
        \item Both computers use this number to encrypt \& decrypt
        \item Patient and doctor can understand each other
        \item For everyone else, the data stream looks like gibberish (random)
        \item Well-made cryptography even protects info about who is communicating
      \end{itemize}
    \end{column}
    \hfill
  \end{columns}
\end{frame}


%\begin{frame}[T]{Kryptografische Sicherheit ist exponentiell}
%  \begin{itemize}
%    \item Eine Einheit mehr Arbeit für ehrliche Parteien
%    \item => Doppelte Arbeit für Angreifer (oder ein anderer Faktor)
%  \end{itemize}
%\end{frame}
%
%\begin{frame}[T]{Exponentialrechnung: Krishna's Rice Pudding}
%  Bild: 
%  \begin{itemize}
%    \item Ambalappuzha Sree Krishna Swamy Temple Legend
%    \item \url{https://en.wikipedia.org/wiki/Ambalappuzha_Sree_Krishna_Swamy_Temple}
%    \item Schachbrett mit figuren
%    \item "Hey, können sie die Weltreisproduktion um den Faktor 1300 erhöhen"
%    \item Höchster turm reicht bis weit hinter pluto (13 licht-tage)
%    \item Planet pluto mit messer und gabel
%  \end{itemize}
%\end{frame}
%
%\begin{frame}[T]{Exponentialrechnung: Weltwirtschaftswachstum}
%  \begin{itemize}
%    \item Angenommen Wirtschaftswachstum von 2.5\%
%    \item Angenommen Energienutzung proportional zum Wirtschaftswachstum
%    \item Erdenergiebudget in 300 Jahren überschritten
%    \item Weitere 600 Jahre: Sonnenenergiebudget überschritten
%    \item Weitere 2500 Jahre: Energiebudget des Universums überschritten
%    \item (Das geht nicht weil Lichtgeschwindigkeit)
%  \end{itemize}
%\end{frame}
%
%\begin{frame}[T]{Exponenzielle Sicherheit}
%  \begin{itemize}
%    \item Kleinstmögliche energiekosten (Landauer Limit) für kostenschritte
%    \item Energiebudget der Erde verbraucht: ~150 Verdopplungsschritte oder 4200 schritte bei faktor 2.5\%
%    \item Energiebudget der Sonne verbraucht: ~180 Verdopplungsschritte oder 5000 bei faktor 2.5\%
%    \item Energieinhalt des Universums verbraucht: ~300 Verdopplungsschritte oder 8500 bei faktor 2.5\%
%  \end{itemize}
%\end{frame}
