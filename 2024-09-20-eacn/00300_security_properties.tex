\interlude[1]{Secure Channels \vfill \Large and their Security Properties}

\begin{frame}{Secure Channel Protocols}
  \begin{columns}[T]
    \begin{column}{.4\linewidth}
      \small

      Secure channel protocols like TLS, OpenSSH, or the Noise Protocol Framework \citeNoise are used everywhere on the internet. They are

      \begin{itemize}
        \item Cheap
        \item Fast
        \item Secure
        \item Well analyzed
        \item Authenticated
        \item Usually not secure against quantum attacks
      \end{itemize}
    \end{column}
    \begin{column}{.55\linewidth}
     \includegraphics[width=\linewidth]{graphics/2024-09-13-rosenpass-tls-cert.png}
    \end{column}
  \end{columns}
\end{frame}

\begin{frame}{Security against quantum attacks}
  \begin{columns}[fullwidth,c]
    \begin{column}{.6\linewidth}
      \includegraphics[height=\defaultframetextheight]{graphics/rosenpass-wp-key-exchange-protocol-rgb.pdf}
    \end{column}

    \begin{column}{.4\linewidth}
    \stretchcolumn{
    \vfill
      \begin{itemize}
        \item Migration to post-quantum security is definetly possible
        \item Rosenpass (pictured) is an example
        \item Alternatives such as KEM-TLS have also been proposed
        \item Some high security protocols such as OpenSSH and the Signal protocol are already using hybrid PQC
        \item Modest increase in resource usage
      \end{itemize}
      \vfill
      }
    \end{column}
  \end{columns}
\end{frame}

\begin{frame}{Active \& Passive Security}
  TODO
\end{frame}

\begin{frame}{Secrecy \& Authenticity}
  TODO
\end{frame}

\begin{frame}{Identity hiding, deniability}
  TODO
\end{frame}

\begin{frame}{Advanced properties}
  Often provided by secure messaging protocols such as Signal or MLS

  \begin{itemize}
    \item Post-compromise security (recovering security after a compromise)
    \item Group messaging
    \item Metadata obfuscation
    \item Asynchronous handshakes (one party offline)
  \end{itemize}
\end{frame}

\begin{frame}{Forward secrecy}
  TODO
\end{frame}

\begin{frame}{Everlasting secrecy}
  TODO
\end{frame}

\begin{frame}{QKD Caveats}
  \begin{columns}
    \begin{column}{.4\textwidth}

      % TODO(marei): Can we use proper checkmark/cross symbols here here?
      % Can we use a light-green background for green, a light-red one for cross and a yellow one for impractical?
      % Can we generally make this look nice and graphic-y?
      \begin{tabular}{ l c c }
        \textbf{Security property} & \textbf{QKD} & \textbf{Software encryption} \\

        Post-Quantum        & Check           & Check           \\
        Forward-secrecy     &                 & Check           \\
        Everlasting-Secrecy & Impractical     & Cross           \\
        End-2-End           & Impractical     & Check           \\
        Active Attackers    & Cross           & Check           \\
        Authenticity        & Cross           & Check           \\
        Deniability         & Cross           & Check           \\
        Non-repudiation     & Cross           & Check           \\
        Identity hiding     & Cross           & Check           \\
      \end{tabular}
    \end{column}
    \begin{column}{.6\textwidth}
      QKD is…

      \begin{itemize}
        \item Expensive
        \item Inefficient
        \item Everlasting secrecy would be nice, but is impractical for real-world setups
        \item Multi-hop security is impractical
        \item End-2-end security is missing entirely (no QKD on my end-user device fiesable for now)
      \end{itemize}
    \end{column}
  \end{columns}
\end{frame}
