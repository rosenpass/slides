\ExplSyntaxOn
\int_gset:Nn \g__ptxcd_interlude_page_int {-1}
\ExplSyntaxOff

\interlude{What do we want?}

\begin{frame}[T]{}
  \centering
  \bsixonetwo

  { \footnotesize What do we want? }
  \\[-0.00cm] \resizebox{10cm}{!}{\bfseries \Large data communication}
  \\[.2cm] { \footnotesize How do we want it? }
  \\[-0.32cm] \resizebox{10cm}{!}{\bfseries \Large securely}
  \\[-0.15cm] { \footnotesize Specifically? }
  \\[.08cm] \resizebox{10cm}{!}{\bfseries \Large post-quantum secure}
\end{frame}

\begin{frame}[T]{}
  \begin{tabular}{lrrr}
                         & \textbf{QKD}                         & \textbf{Bricks} & \textbf{Crypto.} \\
    End to End Security  & No                                   & Yes        & Yes          \\
    Authentication       & Yes\footnote{Through Wegman-Carter}  & Yes        & Yes          \\
    Commodity Hardware   & No                                   & No         & Yes          \\
    Data rates           & kilobits                             & Arbitrary  & Arbitrary    \\
    Information-theoretic security\footnote{Everlasting Secrecy, the lack of algorithmic hardness assumptions}
      & (No)\footnote{Not with these data rates}
      & (Yes)\footnote{Assuming pressure hammer attacks are detected}
      & (Yes)\footnote{With a suitcase of hard drives containing keys} 
  \end{tabular}
\end{frame}

\begin{frame}[T]{}
  \centering
  \bsixonetwo
  { \small How to secure the internet against quantum attacks? }
  \\[0.3cm] \resizebox{14.3cm}{!}{\bfseries \Large With computational cryptography}
\end{frame}

\interlude{How do we think about QKD then?}

\begin{frame}[T]{}
  \centering
  \includegraphics[height=1.15\defaultframetextheight,clip,trim=0 25 0 65]{scientific/rosenpass-qkd.pdf}
  \\ \textbf{QKD:} A fail-over in case Post-Quantum Cryptography fails.
\end{frame}

\begin{frame}[T]{}
  \centering
  PLACEHOLDER
  \\ \textbf{QKD:} A measure of hardware security.
\end{frame}

\begin{frame}[T]{}
  \centering
  \bsixonetwo

  { \footnotesize What do we want? }
  \\[-0.00cm] \resizebox{10cm}{!}{\bfseries \Large secure data communication}
  \\[.08cm] { \footnotesize Where do we want it? }
  \\[.2cm] \resizebox{10cm}{!}{\bfseries \Large on highly secure institutional networks}
  \\[.0cm] { \footnotesize In what particular manner? }
  \\[.00cm] \resizebox{10cm}{!}{\bfseries \Large with hardware security measures}
  \\[.08cm] \resizebox{10cm}{!}{\bfseries \Large especially QKD}
  \\[.1cm] { \footnotesize Anything else else? }
  \\[.1cm] \resizebox{10cm}{!}{\bfseries \Large interoperability with the internet}
\end{frame}

\ExplSyntaxOn
\int_gset:Nn \g__ptxcd_interlude_page_int {-1}
\ExplSyntaxOff

\interlude{How do we build it?}

\begin{frame}[T]{How about a key management system?}
  \begin{columns}[T,fullwidth]
    \hfill
    \begin{column}{.55\linewidth}
      \includegraphics[height=\defaultframetextheight,trim=65 0 65 0,clip]{comic/rosenpass-comic-rgm-2.png}
    \end{column}
    \begin{column}{.38\linewidth}
      \vspace{0.8em}
      \begin{itemize}
        \item Pretty expensive
        \item Pretty complicated
        \item Still requires IP-based networking
        \item One compromised node compromises the entire network
        \item Does not address how to create data channels
      \end{itemize}
    \end{column}
    \hfill
  \end{columns}
\end{frame}

\begin{frame}[T]{}
  \centering
  \includegraphics[height=\defaultframetextheight]{comic/rosenpass-comic-rgm-3.png}
  \\ Key management systems evoke the image of a rube-goldberg machine.
\end{frame}

\begin{frame}[T]{}
  \centering
  PLACEHOLDER \\
  The internet is an architecture for \textbf{transport-agnostic} networking.
\end{frame}

\begin{frame}[T]{}
  \centering
  PLACEHOLDER \\
  Then QKD is just another transport technology, with extra security features.
\end{frame}

\ExplSyntaxOn
\int_gset:Nn \g__ptxcd_interlude_page_int {-1}
\ExplSyntaxOff

\interlude{That's a wrap, problem solved!}

\begin{frame}[T]{}
  \begin{columns}[T,fullwidth]
    \hfill
    \begin{column}{.55\linewidth}
      \centering
      \includegraphics[height=\defaultframetextheight]{comic/rosenpass-comic-rgm-2.png}
      \\ Doing this securely means we need secure routing.
    \end{column}
    \begin{column}{.38\linewidth}
      \vspace{0.8em}
      With internet standard technologies:
      \begin{itemize}
        \item \textbf{SRv6} to fully control the routes packages take \\ …even if nodes not on the path are compromised.
        \item \textbf{HNCP} to learn the network topology \\ …and to automatically deploy networks in the first place.
      \end{itemize}
    \end{column}
    \hfill
  \end{columns}
\end{frame}

\begin{frame}[T]{}
  \centering
  \includegraphics[height=\defaultframetextheight]{comic/rosenpass-comic-rgm-2.png}
  \\ And we need end-to-end hybrid computational security.
  \\ For instance using Rosenpass for post-quantum security and WireGuard for classical security.
\end{frame}

\begin{frame}[T]{}
  \centering
  \includegraphics[height=\defaultframetextheight]{comic/rosenpass-comic-rgm-2.png}
  \\ Ingress to egress security can substitute for end to end security in corporate environments
  \\ …so the technology does not have to be installed on every old windows laptop.
\end{frame}

\begin{frame}[T]{}
  \begin{columns}[T,fullwidth]
    \hfill
    \begin{column}{.55\linewidth}
      \centering
      \includegraphics[height=\defaultframetextheight]{comic/rosenpass-comic-rgm-2.png}
      \\ Comparing the architectures
    \end{column}
    \begin{column}{.38\linewidth}
      \vspace{0.8em}
      With internet standard technologies:
      \begin{itemize}
        \item QKD becomes just another transport
        \item KMS replaced with HNCP (observability) and SRv6 (secure routing)
        \item Management application package ties this into a neat bundle
      \end{itemize}
      Additional features:
      \begin{itemize}
        \item Interoperability with the normal internet
        \item Hardware security measures other than QKD supported
        \item Automatic network deployment through HNCP
      \end{itemize}
    \end{column}
    \hfill
  \end{columns}
\end{frame}

\ExplSyntaxOn
\int_gset:Nn \g__ptxcd_interlude_page_int {-1}
\ExplSyntaxOff

\interlude{This is in fact a Key Management System}

\begin{frame}[T]{}
  \centering
  \includegraphics[height=\defaultframetextheight]{comic/rosenpass-comic-rgm-2.png}
  \\ Keys sent on a secured path, gain the security properties of the secured path.
  \\ So we can implement a KMS, if we really want to, by exposing an API that chooses a random key, then transmits it.
\end{frame}

\begin{frame}[T]{}
  \centering
  \includegraphics[height=\defaultframetextheight]{comic/rosenpass-comic-rgm-2.png}
  \\ \textbf{Information theoretic security:} supported if the transports do support it.
\end{frame}

\begin{frame}[T]{}
  \centering
  \includegraphics[height=\defaultframetextheight]{comic/rosenpass-comic-rgm-2.png}
  \\ Quantum repeaters: Just a special type of transport, no distinction for the network.
\end{frame}

\interlude{What about a proper QKD-enabled internet?}

\begin{frame}[T]{}
  \centering
  \includegraphics[height=\defaultframetextheight]{comic/rosenpass-comic-rgm-2.png}
  \\ \textbf{Do not reinvent the wheel}, use established routing protocols:
  \\ Build an extension to IPv6, that can transmit QKD keys alongside packages.
\end{frame}

\begin{frame}[T]{}
  \centering
  \includegraphics[height=\defaultframetextheight]{comic/rosenpass-comic-rgm-2.png}
  We are collaborating with Quantum Optics Jena to realize this
\end{frame}

\ExplSyntaxOn
\int_gset:Nn \g__ptxcd_interlude_page_int {-1}
\ExplSyntaxOff

\interlude{Key takeaways}

\begin{frame}[T]{}
  \begin{columns}[T,fullwidth]
    \hfill
    \begin{column}{.55\linewidth}
      \centering
      \includegraphics[height=\defaultframetextheight]{comic/rosenpass-comic-rgm-2.png}
    \end{column}
    \begin{column}{.38\linewidth}
      \vspace{0.8em}
      Key takeaways:
      \begin{itemize}
        \item QKD is a hardware security measure, not a replacement for cryptography
        \item Open Source, Open Source, Open Source
        \item We can use or extend standard internet technologies, to build the quantum internet
      \end{itemize}
    \end{column}
    \hfill
  \end{columns}
\end{frame}
